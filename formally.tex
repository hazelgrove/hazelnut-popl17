% !TEX root = hazelnut-popl17.tex

The previous section introduced Hazelnut by example. In this section, we
systematically introduce the following structures:
\begin{itemize}[itemsep=0px,partopsep=2px,topsep=2px]
\item \textbf{H-types} and \textbf{H-expressions} (Sec. \ref{sec:holes}),
  which are types and expressions with {holes}. H-types classify
  H-expressions according to Hazelnut's \textbf{bidirectional static
    semantics}.

\item \textbf{Z-types} and \textbf{Z-expressions} (Sec. \ref{sec:cursors}),
  which superimpose\- a \emph{cursor} onto H-types and H-expressions,
  respectively (following Huet's \emph{zipper pattern}
  \cite{JFP::Huet1997}.) Every Z-type (resp. Z-expression) corresponds to
  an H-type (resp. H-expression) by \emph{cursor erasure}.

\item \textbf{Actions} (Sec. \ref{sec:actions}), which act relative to the
  cursor according to Hazelnut's \textbf{bidirectional action
    semantics}. The action semantics enjoys a rich metatheory. Of
  particular note, the \emph{sensibility theorem} establishes that every
  edit state is well-typed after cursor erasure.
\end{itemize}

Our overview below omits certain ``uninteresting'' details. The supplement
includes the complete collection of rules, in definitional order. These
rules, along with the proofs of two key metatheorems, have been mechanized
in Agda \cite{norell:thesis}, also in the supplement. We will summarize
this effort in Sec. \ref{sec:mech}.

\subsection{H-types and H-expressions}\label{sec:holes}
Figure \ref{fig:hexp-syntax} defines the syntax of H-types, $\htau$, and
H-expressions, $\hexp$. Most forms correspond directly to those of the
simply typed lambda calculus (STLC) extended with a single base type,
$\tnum$, of numbers (cf. \cite{pfpl}.) The number expression corresponding
to the mathematical number $n$ is drawn $\hnum{n}$, and for simplicity, we
define only a single arithmetic operation, $\hadd{\hexp}{\hexp}$. The form
$\hexp : \htau$ is an explicit \emph{type ascription}.

In addition to these standard forms, \emph{type holes} and \emph{empty
  expression holes} are both drawn $\hehole$ and \emph{non-empty expression
  holes} are drawn $\hhole{\hexp}$. We do not need non-empty type holes
because every H-type is a valid classifier of H-expressions.

Types and expressions that contain no holes are \emph{complete types} and
\emph{complete expressions}, respectively. Formally, we can derive
$\hcomplete{\htau}$ when $\htau$ is complete, and $\hcomplete{\hexp}$ when
$\hexp$ is complete (see supplement.) We are not concerned here with
defining a dynamic semantics for Hazelnut, but the dynamics for complete
H-expressions would be entirely standard (in Sec. \ref{sec:rw} we discuss
the multi-dimensional design space around the question of evaluating
incomplete H-expressions.)

\begin{figure}[t]
$\arraycolsep=4pt\begin{array}{lllllll}
\mathsf{HTyp} & \htau & ::= &
  \tarr{\htau}{\htau} ~\vert~
  \tnum ~\vert~
  \tehole\\
\mathsf{HExp} & \hexp & ::= &
  \hexp : \htau ~\vert~
  x ~\vert~
  \hlam{x}{\hexp} ~\vert~
  \hap{\hexp}{\hexp} ~\vert~
  \hnum{n} ~\vert~
  \hadd{\hexp}{\hexp} ~\vert~
  \hehole ~\vert~
  \hhole{\hexp}
\end{array}$
\caption{Syntax of H-types and H-expressions. Metavariable $x$ ranges over variables and $n$ ranges over numerals.}
\label{fig:hexp-syntax}
\end{figure}
\begin{figure}
\noindent\fbox{$\tcompat{\htau}{\htau'}$}~~\text{$\tau$ and $\tau'$ are consistent}
\begin{subequations}\label{eqns:consistency}
\begin{mathpar}

\inferrule{ }{
  \tcompat{\tehole}{\htau}
}
~~~~~~~~
\inferrule{ }{
  \tcompat{\htau}{\tehole}
}
~~~~~~~~
\inferrule{ }{
  \tcompat{\htau}{\htau}
}
~~~~~~~~
\inferrule{
  \tcompat{\htau_1}{\htau_1'}\\
  \tcompat{\htau_2}{\htau_2'}
}{
  \tcompat{(\tarr{\htau_1}{\htau_2})}{(\tarr{\htau_1'}{\htau_2'})}
}~~~~~~~~\text{(\ref*{eqns:consistency}a-d)}
\end{mathpar}
\end{subequations}
\fbox{$\arrmatch{\htau}{\tarr{\htau_1}{\htau_2}}$}~~\text{$\tau$ has matched arrow type $\tarr{\htau_1}{\htau_2}$}
\begin{subequations}
\begin{minipage}{0.5\linewidth}
\begin{equation}
\inferrule{ }{
  \arrmatch{\tarr{\htau_1}{\htau_2}}{\tarr{\htau_1}{\htau_2}}
}
\end{equation}
\end{minipage}
\begin{minipage}{0.5\linewidth}
\begin{equation}
\inferrule{ }{
  \arrmatch{\tehole}{\tarr{\tehole}{\tehole}}
}
\end{equation}
\end{minipage}
\end{subequations}
\caption{H-type consistency and matched arrow types.}
\label{fig:type-consistency}
\end{figure}

\begin{figure}
\fbox{$\hsyn{\hGamma}{\hexp}{\htau}$}~~\text{$\hexp$ synthesizes $\htau$}
\begin{subequations}
\begin{equation}\label{rule:syn-asc}
\inferrule{
  \hana{\hGamma}{\hexp}{\htau}
}{
  \hsyn{\hGamma}{\hexp : \htau}{\htau}
}
\end{equation}
\begin{equation}\label{rule:syn-var}
\inferrule{ }{
  \hsyn{\hGamma, x : \htau}{x}{\htau}
}
\end{equation}
\begin{equation}\label{rule:syn-ap}
\inferrule{
  \hsyn{\hGamma}{\hexp_1}{\htau}\\
  \arrmatch{\htau}{\tarr{\htau_2}{\htau'}}\\
  \hana{\hGamma}{\hexp_2}{\htau_2}
}{
  \hsyn{\hGamma}{\hap{\hexp_1}{\hexp_2}}{\htau'}
}
\end{equation}
\begin{equation}\label{rule:syn-num}
\inferrule{ }{
  \hsyn{\hGamma}{\hnum{n}}{\tnum}
}
\end{equation}
\begin{equation}\label{rule:syn-plus}
\inferrule{
  \hana{\hGamma}{\hexp_1}{\tnum}\\
  \hana{\hGamma}{\hexp_2}{\tnum}
}{
  \hsyn{\hGamma}{\hadd{\hexp_1}{\hexp_2}}{\tnum}
}
\end{equation}
\begin{equation}\label{rule:syn-ehole}
\inferrule{ }{
  \hsyn{\hGamma}{\hehole}{\tehole}
}
\end{equation}
\begin{equation}\label{rule:syn-hole}
\inferrule{
  \hsyn{\hGamma}{\hexp}{\htau}
}{
  \hsyn{\hGamma}{\hhole{\hexp}}{\tehole}
}
\end{equation}
\end{subequations}
\fbox{$\hana{\hGamma}{\hexp}{\htau}$}~~\text{$\hexp$ analyzes against $\htau$}
\begin{subequations}
\begin{equation}\label{rule:ana-subsume}
\inferrule{
  \hsyn{\hGamma}{\hexp}{\htau'}\\
  \tcompat{\htau}{\htau'}
}{
  \hana{\hGamma}{\hexp}{\htau}
}
\end{equation}
\begin{equation}\label{rule:syn-lam}
\inferrule{
  \arrmatch{\htau}{\tarr{\htau_1}{\htau_2}}\\
  \hana{\hGamma, x : \htau_1}{\hexp}{\htau_2}
}{
  \hana{\hGamma}{\hlam{x}{\hexp}}{\htau}
}
\end{equation}
\end{subequations}
\caption{Synthesis and analysis.}
\label{fig:ana-syn}
\end{figure}

Hazelnut's static semantics is organized as a \emph{bidirectional type
  system}
\cite{Pierce:2000:LTI:345099.345100,DBLP:conf/icfp/DaviesP00,DBLP:conf/tldi/ChlipalaPH05,bidi-tutorial}
around the two mutually defined judgements in Figure
\ref{fig:ana-syn}. Typing contexts, $\hGamma$, map each variable $x \in
\domof{\hGamma}$ to a hypothesis $x : \htau$ in the usual manner. We
identify the context up to exchange and adopt the standard identification
convention for formal structures that differ only up to alpha-equivalence.

Derivations of the type analysis judgement, $\hana{\hGamma}{\hexp}{\htau}$,
establish that $\hexp$ can appear where an expression of type $\htau$ is
expected. Derivations of the type synthesis judgement,
$\hsyn{\hGamma}{\hexp}{\htau}$, infer a type from $\hexp$, which is
necessary in positions where an expected type is not available (e.g. at the
top level.) Algorithmically, the type is an ``input'' of the type analysis
judgement, but an ``output'' of the type synthesis judgement.  Making a
judgemental distinction between these two notions will be essential in our
action semantics (Sec. \ref{sec:actions}.)

\begin{subequations}\label{rules:syn-ana}
Type synthesis is stronger than type analysis in that if an expression is
able to synthesize a type, it can also be analyzed against that type, or
any other \emph{consistent} type, according to the \emph{subsumption rule},
Rule (\ref{rule:ana-subsume}).

The \emph{H-type consistency judgement}, $\tcompat{\htau}{\htau'}$, that
appears as a premise in the subsumption rule is a reflexive and symmetric
(but not transitive) relation between H-types defined judgementally in
Figure \ref{fig:type-consistency}. This relation coincides with equality
for complete H-types. Two incomplete H-types are consistent if they differ
only at positions where a hole appears in either type. The type hole is
therefore consistent with every type. This notion of H-type consistency
coincides with the notion of type consistency that Siek and Taha discovered
in their foundational work on gradual type systems, if we interpret the
type hole as the $?$ (i.e. unknown) type \cite{Siek06a}.

Rule (\ref{rule:syn-lam}) defines analysis for lambda abstractions,
$\hlam{x}{\hexp}$. There is no type synthesis rule that applies to this
form, so lambda abstractions can appear only in analytic position,
i.e. where an expected type is known.\footnote{It is possible to also
  define a ``half-annotated'' synthetic lambda form, $\lambda x{:}\tau.e$,
  but for simplicity, we leave it out \cite{DBLP:conf/tldi/ChlipalaPH05}.}
Rule (\ref{rule:syn-lam}) is not quite the standard rule, reproduced below:
\begin{equation*}
\inferrule{
  \hana{\hGamma, x : \htau_1}{\hexp}{\htau_2}
}{
  \hana{\hGamma}{\hlam{x}{\hexp}}{\tarr{\htau_1}{\htau_2}}
}
\end{equation*}
The problem is that this standard rule alone leaves us unable to analyze
lambda abstractions against the type hole, because the type hole is not
immediately of the form $\tarr{\htau_1}{\htau_2}$. There are two plausible
solutions to this problem. One solution would be to define a second rule
specifically for this case:
\begin{equation*}
\inferrule{
  \hana{\hGamma, x : \tehole}{\hexp}{\tehole}
}{
  \hana{\hGamma}{\hlam{x}{\hexp}}{\tehole}
}
\end{equation*}
Instead, we combine these two possible rules into a single rule through the
simple auxiliary \emph{matched arrow type} judgement,
$\arrmatch{\htau}{\tarr{\htau_1}{\htau_2}}$, defined in Figure
\ref{fig:type-consistency}. This judgement leaves arrow types alone and
assigns the type hole the matched arrow type $\tarr{\tehole}{\tehole}$. It
is easy to see that the two rules above are admissible by appeal to Rule
(\ref{rule:syn-lam}) and the matched arrow type judgement. Encouragingly,
this judgement also arises in the study of gradual type systems
\cite{DBLP:conf/popl/CiminiS16,DBLP:conf/popl/GarciaC15,DBLP:conf/popl/RastogiCH12}.

Rule (\ref{rule:syn-asc}) defines type synthesis of expressions of
ascription form, $\hexp : \htau$. This allows the user to explicitly state
a type for the ascribed expression to be analyzed against.

Rule (\ref{rule:syn-var}) is the standard rule for variables.

Rule (\ref{rule:syn-ap}) is again nearly the standard rule for function
application. It also makes use of the matched function type judgement to
combine what would otherwise need to be two rules for function application
-- one for when $e_1$ synthesizes an arrow type, and another for when $e_1$
synthesizes the type hole. Indeed, Siek and Taha needed two application
rules for the same fundamental reason \cite{Siek06a}. Later work on gradual
typing introduced this notion of type matching to resolve this redundancy.

Rule (\ref{rule:syn-num}) states that numbers synthesize the $\tnum$
type. Rule (\ref{rule:syn-plus}) states that $\hexp_1 + \hexp_2$ behaves
like a function over numbers.

The rules described so far are sufficient to type complete
H-expressions. The two remaining rules give H-expressions with holes a
well-defined static semantics.

Rule (\ref{rule:syn-ehole}) states that the empty expression hole
synthesizes the type hole.

A non-empty hole contains an H-expression that is ``under construction'',
as described in Sec. \ref{sec:example}. According to Rule
(\ref{rule:syn-hole}), this inner expression must synthesize some type,
but, like the empty hole, non-empty holes synthesize the type hole.

Given these rules, it is instructive to derive the following (by subsumption):
\[\hana{incr : \tarr{\tnum}{\tnum}}{\hhole{incr}}{\tnum}\]

\end{subequations}
\subsection{Z-types and Z-expressions}\label{sec:cursors}
\newcommand{\cvert}{{\,{\vert}\,}}
\begin{figure}[t]
\hspace{-3px}$\arraycolsep=2pt\begin{array}{lllllll}
\mathsf{ZTyp} & \ztau & ::= &
  \zwsel{\htau} \cvert
  \tarr{\ztau}{\htau} \cvert
  \tarr{\htau}{\ztau} \\
\mathsf{ZExp} & \zexp & ::= &
  \zwsel{\hexp} \cvert
  \zexp : \htau \cvert
  \hexp : \ztau \cvert
  \hlam{x}{\zexp} \cvert
  \hap{\zexp}{\hexp} \cvert
  \hap{\hexp}{\zexp} \cvert
  \hadd{\zexp}{\hexp} \cvert
  \hadd{\hexp}{\zexp} \cvert
  \hhole{\zexp}
\end{array}$
\caption{Syntax of Z-types and Z-expressions.}
\label{fig:zexp-syntax}
\end{figure}

Hazelnut's action semantics operates over Z-types, $\ztau$, and
Z-expressions, $\zexp$. Figure \ref{fig:zexp-syntax} defines the syntax of
Z-types and Z-expressions. A Z-type (resp. Z-expression) represents an
H-type (resp. H-expression) with a single superimposed \emph{cursor}.

The only base cases in these inductive grammars are $\zwsel{\htau}$ and
$\zwsel{\hexp}$, which identify the H-type or H-expression that the cursor
is on. All of the other forms correspond to the recursive forms in the
syntax of H-types and H-expressions, and contain exactly one ``hatted''
subterm that identifies the subtree where the cursor will be found. Any
other sub-term is ``dotted'', i.e. it is either an H-type or
H-expression. Taken together, every Z-type and Z-expression contains
exactly one selected H-type or H-expression by construction. This can be
understood as an instance of Huet's \emph{zipper pattern}
\cite{JFP::Huet1997} (which, coincidentally, Huet encountered while
implementing a structure editor.)

We write $\removeSel{\ztau}$ for the H-type constructed by erasing the
cursor from $\ztau$, which we refer to as the \emph{cursor erasure} of
$\ztau$. This straightforward metafunction is defined as follows:
\begin{align*}
\removeSel{(\zwsel{\htau})} & = \htau\\
\removeSel{(\tarr{\ztau}{\htau})} & = \tarr{\removeSel{\ztau}}{\htau}\\
\removeSel{(\tarr{\htau}{\ztau})} & = \tarr{\htau}{\removeSel{\ztau}}
\end{align*}

Similarly, we write $\removeSel{\zexp}$ for the H-expression constructed by
erasing the cursor from the Z-expression $\zexp$, i.e. the cursor erasure
of $\zexp$. The definition of this metafunction is analogous, so we omit it
for concision (see supplement.)

\subsection{Actions}\label{sec:actions}

We now arrive at the heart of Hazelnut: its \emph{bidirectional action
  semantics}.  Figure \ref{fig:action-syntax} defines the syntax of
\emph{actions}, $\alpha$, some of which involve \emph{directions},
$\delta$, and \emph{shapes}, $\psi$.

Expression actions are governed by two mutually defined judgements, 1) the
\emph{synthetic action judgement}:
\[
\performSyn{\hGamma}{\zexp}{\htau}{\alpha}{\zexp'}{\htau'}
\]
and 2) \emph{the analytic action judgement}:
\[
\performAna{\hGamma}{\zexp}{\htau}{\alpha}{\zexp'}
\]

In some Z-expressions, the cursor is in a type ascription, so we also need
a \emph{type action judgement}, pronounced ``performing $\alpha$ on $\ztau$
results in $\ztau'$'':
\[
\performTyp{\ztau}{\alpha}{\ztau'}
\]

\subsubsection{Sensibility}


Before giving the rules for these judgements, let us state the key
metatheorem: \emph{sensibility}. This metatheorem deeply informs the design
of the rules, given starting in Sec. \ref{sec:action-subsumption}. Its
proof, which is mechanized in Agda, is by straightforward induction, so the
reader is encouraged to think about the relevant cases of these proofs as
we present the rules.
\begin{theorem}[Action Sensibility]
  \label{thrm:actsafe} Both of the following hold:
  \begin{enumerate}[itemsep=0px,partopsep=0px,topsep=0px]
  \item If $\hsyn{\hGamma}{\removeSel{\zexp}}{\htau}$ and
    $\performSyn{\hGamma}{\zexp}{\htau}{\alpha}{\zexp'}{\htau'}$ then
    $\hsyn{\hGamma}{\removeSel{\zexp'}}{\htau'}$.
  \item If $\hana{\hGamma}{\removeSel{\zexp}}{\htau}$ and
    $\performAna{\hGamma}{\zexp}{\htau}{\alpha}{\zexp'}$ then
    $\hana{\hGamma}{\removeSel{\zexp'}}{\htau}$.
  \end{enumerate}
\end{theorem}
\noindent In other words, if an edit state (i.e. a Z-expression) is
statically meaningful, i.e. its cursor erasure is well-typed, then
performing an action on it leaves the resulting edit state statically
meaningful. In particular, the first clause of Theorem \ref{thrm:actsafe}
establishes that when an action is performed on an edit state whose cursor
erasure synthesizes an H-type, the result is an edit state whose cursor
erasure also synthesizes some (possibly different) H-type. The second
clause establishes that when an action is performed using the analytic
action judgement on an edit state whose cursor erasure analyzes against
some H-type, the result is a Z-expression whose cursor erasure also
analyzes against the same H-type.

\begin{figure}[t]
\hspace{-3px}$\arraycolsep=3pt\begin{array}{llcllll}
\mathsf{Action} & \alpha & ::= &
  \aMove{\delta} ~\vert~
  \aConstruct{\psi} ~\vert~
  \aDel ~\vert~
  \aFinish\\
\mathsf{Dir} & \delta & ::= &
  \dChildnm{n} ~\vert~
  \dParent\\
\mathsf{Shape} & \psi & ::= &
  \farr ~\vert~
  \fnum \\
& & \vert &
  \fasc ~\vert~
  \fvar{x} ~\vert~
  \flam{x} ~\vert~
  \fap ~\vert~
  \farg ~\vert~
  \fnumlit{n} ~\vert~
  \fplus\\
& & \vert &
  {\color{gray}\fnehole}
\end{array}$
\caption{Syntax of actions.}
\label{fig:action-syntax}
\end{figure}

\subsubsection{Type Inconsistency}
In some of the rules below, we will need to supplement our definition of
type consistency from Figure \ref{fig:type-consistency} with a definition
of \emph{type inconsistency}, written $\tincompat{\htau}{\htau'}$. One can
define this notion either directly as the metatheoretic negation of type
consistency, or as a separate inductively defined judgement. The supplement
does the latter. The key rule establishes that arrow types are inconsistent
with $\tnum$:
  \begin{equation*}
    \inferrule{ }{
      \tincompat{\tnum}{\tarr{\htau_1}{\htau_2}}
    }
  \end{equation*}
The mechanization proves that this judgemental definition of type
inconsistency is indeed the negation of type consistency.

\subsubsection{Action Subsumption}\label{sec:action-subsumption}

The action semantics includes a subsumption rule similar to the subsumption
rule, Rule (\ref{rule:ana-subsume}), in the statics:
\begin{equation}\label{rule:action-subsume}
  \inferrule{
    \hsyn{\hGamma}{\removeSel{\zexp}}{\htau'}\\
    \performSyn{\hGamma}{\zexp}{\htau'}{\alpha}{\zexp'}{\htau''}\\
    \tcompat{\htau}{\htau''}
  }{
    \performAna{\hGamma}{\zexp}{\htau}{\alpha}{\zexp'}
  }
\end{equation}
In other words, if the cursor erasure of the edit state synthesizes a type,
then we defer to the synthetic action judgement, as long as the type of the
resulting cursor erasure is consistent with the type provided for analysis.
The case involving Rule (\ref{rule:action-subsume}) in the proof of Theorem
\ref{thrm:actsafe} goes through by induction and static subsumption,
i.e. Rule (\ref{rule:ana-subsume}). Algorithmically, subsumption should be
the rule of last resort.

\subsubsection{Relative Movement}\label{sec:movement}
The rules below define relative movement within Z-types. They should be
self-explanatory:
\begin{subequations}
\begin{equation}
  \inferrule{ }{
    \performTyp{
      \zwsel{\tarr{\htau_1}{\htau_2}}
    }{
      \aMove{\dChildn{1}}
    }{
      \tarr{\zwsel{\htau_1}}{\htau_2}
    }
  }
\end{equation}
\begin{equation}\label{rule:move-arr-c2}
  \inferrule{ }{
    \performTyp{
      \zwsel{\tarr{\htau_1}{\htau_2}}
    }{
      \aMove{\dChildn{2}}
    }{
      \tarr{\htau_1}{\zwsel{\htau_2}}
    }
  }
\end{equation}
\begin{equation}\label{rule:move-parent-arr-left}
  \inferrule{ }{
    \performTyp{
      \tarr{\zwsel{\htau_1}}{\htau_2}
    }{
      \aMove{\dParent}
    }{
      \zwsel{\tarr{\htau_1}{\htau_2}}
    }
  }
\end{equation}
\begin{equation}\label{rule:move-parent-arr-right}
  \inferrule{ }{
    \performTyp{
      \tarr{{\htau_1}}{\zwsel{\htau_2}}
    }{
      \aMove{\dParent}
    }{
      \zwsel{\tarr{\htau_1}{\htau_2}}
    }
  }
\end{equation}
\end{subequations}
Two more rules are needed to recurse into the zipper structure. We define
these zipper rules in an action-independent manner in
Sec. \ref{sec:zipper-cases}.

The rules for relative movement within Z-expressions are similarly
straightforward. Movement is type-independent, so we defer to an auxiliary
expression movement judgement in both the analytic and synthetic case:
\begin{subequations}
\begin{equation}
\inferrule{
  \performMove{\zexp}{\aMove{\delta}}{\zexp'}
}{
  \performSyn{\hGamma}{\zexp}{\htau}{\aMove{\delta}}{\zexp'}{\htau}
}
\end{equation}
\begin{equation}
  \inferrule{
  \performMove{\zexp}{\aMove{\delta}}{\zexp'}
}{
  \performAna{\hGamma}{\zexp}{\htau}{\aMove{\delta}}{\zexp'}
}
\end{equation}
\end{subequations}
The expression movement judgement is defined as follows.

\paragraph{Ascription}
\begin{subequations}
  \begin{equation}
    \label{r:movefirstchild-asc}
  \inferrule{ }{
    \performTyp{
      \zwsel{\hexp : \htau}
    }{
      \aMove{\dChildn{1}}
    }{
      \zwsel{\hexp} : \htau
    }
  }
  \end{equation}
  \begin{equation}
    \label{r:movesecondchild-asc}
    \inferrule{ }{
    \performTyp{
      \zwsel{\hexp : \htau}
    }{
      \aMove{\dChildn{2}}
    }{
      \hexp : \zwsel{\htau}
    }
  }
\end{equation}
\begin{equation}
  \label{r:moveparent}
  \inferrule{ }{
    \performTyp{
      \zwsel{\hexp} : \htau
    }{
      \aMove{\dParent}
    }{
      \zwsel{\hexp : \htau}
    }
  }
\end{equation}
\begin{equation}\label{rule:move-parent-asc-right}
  \inferrule{ }{
    \performTyp{
      \hexp : \zwsel{\htau}
    }{
      \aMove{\dParent}
    }{
      \zwsel{\hexp : \htau}
    }
  }
\end{equation}

\paragraph{Lambda}\vspace{-3px}
\begin{equation}\label{r:movefirstchild-lam}
\inferrule{ }{
  \performMove{
    \zwsel{\hlam{x}{\hexp}}
  }{
    \aMove{\dChildn{1}}
  }{
    \hlam{x}{\zwsel{\hexp}}
  }
}
\end{equation}
\begin{equation}
  \inferrule{ }{
    \performMove{
      \hlam{x}{\zwsel{\hexp}}
    }{
      \aMove{\dParent}
    }{
      \zwsel{\hlam{x}{\hexp}}
    }
  }
\end{equation}
\paragraph{Application}\vspace{-5px}
\begin{equation}
  \inferrule{ }{
    \performMove{
      \zwsel{\hap{\hexp_1}{\hexp_2}}
    }{
      \aMove{\dChildn{1}}
    }{
      \hap{\zwsel{\hexp_1}}{\hexp_2}
    }
  }
\end{equation}
\begin{equation}
  \inferrule{ }{
    \performMove{
      \zwsel{\hap{\hexp_1}{\hexp_2}}
    }{
      \aMove{\dChildn{2}}
    }{
      \hap{\hexp_1}{\zwsel{\hexp_2}}
    }
  }
\end{equation}
\begin{equation}
  \inferrule{ }{
    \performMove{
      \hap{\zwsel{\hexp_1}}{\hexp_2}
    }{
      \aMove{\dParent}
    }{
      \zwsel{\hap{\hexp_1}{\hexp_2}}
    }
  }
\end{equation}
\begin{equation}\label{r:moveparent-ap2}
  \inferrule{ }{
    \performMove{
      \hap{{\hexp_1}}{\zwsel{\hexp_2}}
    }{
      \aMove{\dParent}
    }{
      \zwsel{\hap{\hexp_1}{\hexp_2}}
    }
  }
\end{equation}

\paragraph{Plus}
\begin{equation}
  \inferrule{ }{
    \performMove{
      \zwsel{\hadd{\hexp_1}{\hexp_2}}
    }{
      \aMove{\dChildn{1}}
    }{
      \hadd{\zwsel{\hexp_1}}{\hexp_2}
    }
  }
\end{equation}
\begin{equation}
  \inferrule{ }{
    \performMove{
      \zwsel{\hadd{\hexp_1}{\hexp_2}}
    }{
      \aMove{\dChildn{2}}
    }{
      \hadd{\hexp_1}{\zwsel{\hexp_2}}
    }
  }
\end{equation}
\begin{equation}
  \inferrule{ }{
    \performMove{
      \hadd{\zwsel{\hexp_1}}{\hexp_2}
    }{
      \aMove{\dParent}
    }{
      \zwsel{\hadd{\hexp_1}{\hexp_2}}
    }
  }
\end{equation}
\begin{equation}
  \inferrule{ }{
    \performMove{
      \hadd{{\hexp_1}}{\zwsel{\hexp_2}}
    }{
      \aMove{\dParent}
    }{
      \zwsel{\hadd{\hexp_1}{\hexp_2}}
    }
  }
\end{equation}

\paragraph{Non-Empty Hole}
\begin{equation}
\inferrule{ }{
  \performMove{
    \zwsel{\hhole{\hexp}}
  }{
    \aMove{\dChildn{1}}
  }{
    \hhole{\zwsel{\hexp}}
  }
}
\end{equation}
\begin{equation}\label{r:moveparent-hole}
  \inferrule{ }{
    \performMove{
      \hhole{\zwsel{\hexp}}
    }{
      \aMove{\dParent}
    }{
      \zwsel{\hhole{\hexp}}
    }
  }
\end{equation}
Again, additional rules are needed to recurse into the zipper structure,
but we will define these zipper rules in an action-independent manner in
Sec. \ref{sec:zipper-cases}.
\end{subequations}

The rules above are numerous and fairly uninteresting. That makes them
quite hazardous -- we might make a mistake absent-mindedly. One check
against this is to prove a lemma that establishes that movement actions
move the cursor but do not change the cursor erasure.

\begin{theorem}[Movement Erasure Invariance]\label{lemma:movement-erasure} ~
  \begin{enumerate}[itemsep=0px,partopsep=0px,topsep=0px]
  \item If $\performMove{\ztau}{\aMove{\delta}}{\ztau'}$ then
    $\removeSel{\ztau}=\removeSel{\ztau'}$.

  \item If $\performMove{\zexp}{\aMove{\delta}}{\zexp'}$ then
    $\removeSel{\zexp}=\removeSel{\zexp'}$.
\end{enumerate}
\end{theorem}
\noindent Theorem \ref{lemma:movement-erasure} is useful also in that the
relevant cases of Theorem \ref{thrm:actsafe} are straightforward by its
application.

Another useful check is to establish \emph{reachability}, i.e. that it is
possible, through a sequence of movement actions, to move the cursor from
any position to any other position within a well-typed H-expression.

This requires developing machinery for reasoning about sequences of
actions. There are two possibilities: we can either add a sequencing
action, $\alpha; \alpha$, directly to the syntax of actions, or we can
define a syntax for lists of actions, $\bar{\alpha}$, together with
iterated action judgements. To keep the core of the action semantics small,
we take the latter approach in Figure \ref{fig:multistep}.

A simple auxiliary judgement, $\bar\alpha~\mathsf{movements}$, defined in
the supplement, establishes that $\bar\alpha$ consists only of actions of
the form $\aMove{\delta}$.

With these definitions, we can state reachability as follows:

\begin{theorem}[Reachability]\label{thrm:reachability} ~
  \begin{enumerate}[itemsep=0px,partopsep=0px,topsep=0px]
  \item If $\removeSel{\ztau}=\removeSel{\ztau'}$ then there exists some
    $\bar\alpha$ such that $\bar{\alpha}~\mathsf{movements}$ and
    $\performTypI{\ztau}{\bar\alpha}{\ztau'}$.

  \item If $\hsyn{\hGamma}{\removeSel{\zexp}}{\htau}$ and
    $\removeSel{\zexp}=\removeSel{\zexp'}$ then there exists some
    $\bar{\alpha}$ such that $\bar{\alpha}~\mathsf{movements}$ and
    $\performSynI{\hGamma}{\zexp}{\htau}{\bar\alpha}{\zexp'}{\htau}$.

  \item If $\hana{\hGamma}{\removeSel{\zexp}}{\htau}$ and
    $\removeSel{\zexp}=\removeSel{\zexp'}$ then there exists some
    $\bar{\alpha}$ such that $\bar{\alpha}~\mathsf{movements}$ and
    $\performAnaI{\hGamma}{\zexp}{\htau}{\bar{\alpha}}{\zexp'}$.
  \end{enumerate}
\end{theorem}

\begin{figure}
$\mathsf{ActionList}$~~$\bar{\alpha} ::= \cdot ~\vert~ \alpha; \bar{\alpha}$\vspace{4px}\\
\fbox{$\performTypI{\ztau}{\bar{\alpha}}{\ztau'}$}

\vspace{-10px}\begin{subequations}
\begin{minipage}{0.35\linewidth}
\begin{equation}
\inferrule{ }{
    \performTypI{\ztau}{\cdot}{\ztau}
}
\end{equation}
\end{minipage}
\begin{minipage}{0.65\linewidth}
\begin{equation}
\inferrule{
  \performTyp{\ztau}{\alpha}{\ztau'}\\
  \performTypI{\ztau'}{\bar{\alpha}}{\ztau''}
}{
  \performTypI{\ztau}{\alpha; \bar{\alpha}}{\ztau''}
}
\end{equation}
\end{minipage}
\end{subequations}

\fbox{$\performSynI{\hGamma}{\zexp}{\htau}{\bar{\alpha}}{\zexp'}{\htau'}$}
\vspace{-10px}

\begin{subequations}\label{rules:iterated-syn}
\hspace{-12px}
\begin{mathpar}
\inferrule{ }{
  \performSynI{\hGamma}{\zexp}{\htau}{\cdot}{\zexp}{\htau}
}
~~~~~~
\inferrule{
  \performSyn{\hGamma}{\zexp}{\htau}{\alpha}{\zexp'}{\htau'}\\\\
  \performSynI{\hGamma}{\zexp'}{\htau'}{\bar{\alpha}}{\zexp''}{\htau''}
}{
  \performSynI{\hGamma}{\zexp}{\htau}{\alpha; \bar{\alpha}}{\zexp''}{\htau''}
}
~~~~~
\text{(\ref*{rules:iterated-syn}a-b)}
\end{mathpar}
\end{subequations}

\fbox{$\performAna{\hGamma}{\zexp}{\htau}{\bar{\alpha}}{\zexp'}$}
\vspace{-12px}
\begin{subequations}\label{rules:iterated-ana}
\begin{mathpar}
~~~~~~~\inferrule{ }{
  \performAnaI{\hGamma}{\zexp}{\htau}{\cdot}{\zexp}
}
~~~~~~~~~~~~~~
\inferrule{
  \performAna{\hGamma}{\zexp}{\htau}{\alpha}{\zexp'}\\\\
  \performAnaI{\hGamma}{\zexp'}{\htau}{\bar\alpha}{\zexp''}
}{
  \performAnaI{\hGamma}{\zexp}{\htau}{\alpha; \bar\alpha}{\zexp''}
}
~~~~~~~~~~
\text{(\ref*{rules:iterated-ana}a-b)}
\end{mathpar}
\end{subequations}
\caption{Iterated Action Judgements}
\label{fig:multistep}
\end{figure}

The simplest way to prove Theorem \ref{thrm:reachability} is to break it
into two lemmas. Lemma \ref{lemma:reach-up} establishes that you can always
move the cursor to the outermost position in an expression. This serves as
a check on our $\aMove{\dParent}$ rules.
\begin{lemma}[Reach Up]\label{lemma:reach-up} ~
  \begin{enumerate}[itemsep=0px,partopsep=0px,topsep=0px]
  \item If $\removeSel{\ztau}=\htau$ then there exists some $\bar\alpha$
    such that $\bar\alpha~\mathsf{movements}$ and
    $\performTypI{\ztau}{\bar\alpha}{\zwsel{\htau}}$.

  \item If $\hsyn{\hGamma}{\hexp}{\htau}$ and $\removeSel{\zexp}=\hexp$
    then there exists some $\bar\alpha$ such that
    $\bar\alpha~\mathsf{movements}$ and
    $\performSynI{\hGamma}{\zexp}{\htau}{\bar\alpha}{\zwsel{\hexp}}{\htau}$.

  \item If $\hana{\hGamma}{\hexp}{\htau}$ and $\removeSel{\zexp}=\hexp$
    then there exists some $\bar\alpha$ such that
    $\bar\alpha~\mathsf{movements}$ and
    $\performAnaI{\hGamma}{\zexp}{\htau}{\bar\alpha}{\zwsel{\hexp}}$.
  \end{enumerate}
\end{lemma}
Lemma \ref{lemma:reach-down} establishes that you can always move the
cursor from the outermost position to any other position. This serves as a
check on our $\aMove{\dChildnm{n}}$ rules.
\begin{lemma}[Reach Down]\label{lemma:reach-down} ~
  \begin{enumerate}[itemsep=0px,partopsep=0px,topsep=0px]
  \item If $\removeSel{\ztau}=\htau$ then there exists some $\bar\alpha$
    such that $\bar\alpha~\mathsf{movements}$ and
    $\performTypI{\zwsel{\htau}}{\bar\alpha}{\ztau}$.

  \item If $\hsyn{\hGamma}{\hexp}{\htau}$ and $\removeSel{\zexp}=\hexp$
    then there exists some $\bar\alpha$ such that
    $\bar\alpha~\mathsf{movements}$ nand
    $\performSynI{\hGamma}{\zwsel{\hexp}}{\htau}{\bar\alpha}{\zexp}{\htau}$.

  \item If $\hana{\hGamma}{\hexp}{\htau}$ and $\removeSel{\zexp}=\hexp$
    then there exists some $\bar\alpha$ such that
    $\bar\alpha~\mathsf{movements}$ and
    $\performAnaI{\hGamma}{\zwsel{\hexp}}{\htau}{\bar\alpha}{\zexp}$.
  \end{enumerate}
\end{lemma}
Theorem \ref{thrm:reachability} follows by straightforward composition of
these two lemmas.

\subsubsection{Construction}\label{sec:construction} The construction
actions, $\aConstruct{\psi}$, are used to construct terms of a shape
indicated by $\psi$ at the cursor.


\paragraph{Types} The $\aConstruct{\farr}$ action constructs an arrow
type. The H-type under the cursor becomes the argument type, and the cursor
is placed on an empty return type hole:
\begin{subequations}
  \begin{equation}
    \label{r:contarr}
  \inferrule{ }{
    \performTyp{
      \zwsel{\htau}
    }{
      \aConstruct{\farr}
    }{
      \tarr{\htau}{\zwsel{\tehole}}
    }
  }
\end{equation}

The $\aConstruct{\fnum}$ action replaces an empty type hole under the
cursor with the $\tnum$ type:
  \begin{equation}
    \label{r:contnum}
  \inferrule{ }{
    \performTyp{
      \zwsel{\tehole}
    }{
      \aConstruct{\fnum}
    }{
      \zwsel{\tnum}
    }
  }
\end{equation}
\end{subequations}

\begin{subequations}

\paragraph{Ascription} The $\aConstruct{\fasc}$ action operates differently
depending on whether the H-expression under the cursor synthesizes a type
or is being analyzed against a type. In the first case, the synthesized
type appears in the ascription:
\begin{equation}
  \label{r:constructasc}
  \inferrule{ }{
    \performSyn{\hGamma}{\zwsel{\hexp}}{\htau}{\aConstruct{\fasc}}{\hexp : \zwsel{\htau}}{\htau}
  }
\end{equation}
In the second case, the type provided for analysis appears in the ascription:
\begin{equation}
  \inferrule{ }{
    \performAna{\hGamma}{\zwsel{\hexp}}{\htau}{\aConstruct{\fasc}}{\hexp : \zwsel{\htau}}
  }
\end{equation}

\paragraph{Variables} The $\aConstruct{\fvar{x}}$ action places the
variable $x$ into an empty hole. If that hole is being asked to synthesize
a type, then the result synthesizes the hypothesized type:
\begin{equation}
  \label{r:conevar}
  \inferrule{ }{
    \performSyn{\hGamma, x : \htau}{\zwsel{\hehole}}{\tehole}{\aConstruct{\fvar{x}}}{\zwsel{x}}{\htau}
  }
\end{equation}
If the hole is being analyzed against a type that is consistent with the
hypothesized type, then the action semantics goes through the {action
  subsumption rule} described in Sec. \ref{sec:action-subsumption}. If the
hole is being analyzed against a type that is inconsistent with the
hypothesized type, $x$ is placed inside a hole:
\begin{equation}
 \label{r:conevar2}
  \inferrule{
    \tincompat{\htau}{\htau'}
  }{
    \performAna{\hGamma, x : \htau'}{\zwsel{\hehole}}{\htau}{\aConstruct{\fvar{x}}}{\hhole{\zwsel{x}}}
  }
\end{equation}
The rule above featured on Line 15 of Figure \ref{fig:second-example}.

\paragraph{Lambdas} The $\aConstruct{\flam{x}}$ action places a lambda
abstraction binding $x$ into an empty hole. If the empty hole is being
asked to synthesize a type, then the result of the action is a lambda
ascribed the type $\tarr{\tehole}{\tehole}$, with the cursor on the
argument type hole:
\begin{equation}
  \label{r:conelamhole}
  \inferrule{ }{
    \performSyn
      {\hGamma}
      {\zwsel{\hehole}}
      {\tehole}
      {\aConstruct{\flam{x}}}
      {\hlam{x}{\hehole} : \tarr{\zwsel{\tehole}}{\tehole}}
      {\tarr{\tehole}{\tehole}}
  }
\end{equation}
The type ascription is necessary because lambda expressions do not
synthesize a type. If the empty hole is being analyzed against a type with
matched arrow type, then no ascription is necessary:
\begin{equation}\label{rule:performAna-lam-1}
  \inferrule{
    \arrmatch{\htau}{\tarr{\htau_1}{\htau_2}}
  }{
    \performAna
      {\hGamma}
      {\zwsel{\hehole}}
      {\htau}
      {\aConstruct{\flam{x}}}
      {\hlam{x}{\zwsel{\hehole}}}
  }
\end{equation}

Finally, if the empty hole is being analyzed against a type that has no
matched arrow type, expressed in the premise as inconsistency with
$\tarr{\tehole}{\tehole}$, then a lambda ascribed the type
$\tarr{\tehole}{\tehole}$ is inserted inside a hole, which defers the type
inconsistency as previously discussed:
\begin{equation}\label{rule:performAna-construct-lam-2}
  \inferrule{
    \tincompat{\htau}{\tarr{\tehole}{\tehole}}
  }{
    \performAna
      {\hGamma}
      {\zwsel{\hehole}}
      {\htau}
      {\aConstruct{\flam{x}}}
      {\hhole{
        \hlam{x}{\hehole} : \tarr{\zwsel{\tehole}}{\tehole}
      }}
  }
\end{equation}

\paragraph{Application} The $\aConstruct{\fap}$ action applies the
expression under the cursor. The following rule handles the case where the
synthesized type has matched function type:
\begin{equation}
  \label{r:coneapfn}
  \inferrule{
    \arrmatch{\htau}{\tarr{\htau_1}{\htau_2}}
  }{
    \performSyn
      {\hGamma}
      {\zwsel{\hexp}}
      {\htau}
      {\aConstruct{\fap}}
      {\hap{\hexp}{\zwsel{\hehole}}}
      {\htau_2}
  }
\end{equation}
If the expression under the cursor synthesizes a type that is inconsistent
with an arrow type, then we must place that expression inside a hole to
maintain Theorem \ref{sec:holes}:
\begin{equation}
  \inferrule{
    \tincompat{\htau}{\tarr{\tehole}{\tehole}}
  }{
    \performSyn
      {\hGamma}
      {\zwsel{\hexp}}
      {\htau}
      {\aConstruct{\fap}}
      {\hap{\hhole{\hexp}}{\zwsel{\hehole}}}
      {\tehole}
  }
\end{equation}

The $\aConstruct{\farg}$ action instead places the expression under the
cursor in the argument position of an application form. Because the
function position is always an empty hole in this situation, we need only a
single rule:
\begin{equation}
  \inferrule{ }{
    \performSyn
      {\hGamma}
      {\zwsel{\hexp}}
      {\htau}
      {\aConstruct{\farg}}
      {\hap{\zwsel{\hehole}}{\hexp}}
      {\tehole}
  }
\end{equation}

\paragraph{Numbers} The $\aConstruct{\fnumlit{n}}$ action replaces an empty
hole with the number expression $\hnum{n}$. If the empty hole is being
asked to synthesize a type, then the rule is straightforward:
\begin{equation}
  \label{r:conenumnum}
  \inferrule{ }{
    \performSyn
      {\hGamma}
      {\zwsel{\hehole}}
      {\tehole}
      {\aConstruct{\fnumlit{n}}}
      {\zwsel{\hnum{n}}}
      {\tnum}
  }
\end{equation}
If the empty hole is being analyzed against a type that is inconsistent
with $\tnum$, then we must place the number expression inside the hole:
\begin{equation}
  \inferrule{
    \tincompat{\htau}{\tnum}
  }{
    \performAna
      {\hGamma}
      {\zwsel{\hehole}}
      {\htau}
      {\aConstruct{\fnumlit{n}}}
      {\hhole{\zwsel{\hnum{n}}}}
  }
\end{equation}

The $\aConstruct{\fplus}$ action constructs a plus expression with the
expression under the cursor as its first argument. If that expression
synthesizes a type consistent with $\tnum$, then the rule is
straightforward:
\begin{equation}\label{rule:construct-plus-compat}
  \inferrule{
    \tcompat{\htau}{\tnum}
  }{
    \performSyn
      {\hGamma}
      {\zwsel{\hexp}}
      {\htau}
      {\aConstruct{\fplus}}
      {\hadd{\hexp}{\zwsel{\hehole}}}
      {\tnum}
  }
\end{equation}
Otherwise, we must place that expression inside a hole:
\begin{equation}
  \inferrule{
    \tincompat{\htau}{\tnum}
  }{
    \performSyn
      {\hGamma}
      {\zwsel{\hexp}}
      {\htau}
      {\aConstruct{\fplus}}
      {\hadd{\hhole{\hexp}}{\zwsel{\hehole}}}
      {\tnum}
  }
\end{equation}

\paragraph{Non-Empty Holes} The final shape is $\fnehole$. This explicitly
places the expression under the cursor inside a hole:
\begin{equation}
\inferrule{ }{
  \performSyn
    {\hGamma}
    {\zwsel{\hexp}}
    {\htau}
    {\aConstruct{\fnehole}}
    {\hhole{\zwsel{\hexp}}}
    {\tehole}
}
\end{equation}\end{subequations}

The $\fnehole$ shape is grayed out in Figure \ref{fig:action-syntax}
because we do not expect the programmer to perform it explicitly -- other
actions automatically insert holes when a type inconsistency would
arise. Its inclusion is mainly to make it easier to state another
``checksum'' theorem: \emph{constructability}

\paragraph{Constructability}
To check that we have defined ``enough'' construct actions, we need to establish that we can start from an empty hole and arrive at any well-typed expression with, for simplicity, the cursor on the outside (Lemma \ref{lemma:reach-down} allows us to then move the cursor anywhere else.) As with reachability, we rely on the iterated action judgements defined in Figure \ref{fig:multistep}.
\begin{theorem}[Constructability]\label{thrm:constructability} ~
  \begin{enumerate}[itemsep=0px,partopsep=0px,topsep=0px]
  \item For every $\htau$ there exists $\bar\alpha$ such that
    $\performTypI{\zwsel{\tehole}}{\bar\alpha}{\zwsel{\htau}}$.

  \item If $\hsyn{\hGamma}{\hexp}{\htau}$ then there exists $\bar\alpha$
    such
    that: $$\performSynI{\hGamma}{\zwsel{\hhole{}}}{\tehole}{\bar\alpha}{\zwsel{\hexp}}{\htau}$$

  \item If $\hana{\hGamma}{\hexp}{\htau}$ then there exists $\bar\alpha$
    such
    that: $$\performAnaI{\hGamma}{\zwsel{\hhole{}}}{\htau}{\bar\alpha}{\zwsel{\hexp}}$$
  \end{enumerate}
\end{theorem}

\subsubsection{Deletion} The $\aDel$ action inserts an empty hole at the
cursor, deleting what was there before.

The type action rule for $\aDel$ is self-explanatory:
\begin{equation}
  \inferrule{ }{
    \performTyp{
      \zwsel{\htau}
    }{
      \aDel
    }{
      \zwsel{\tehole}
    }
  }
\end{equation}

Deletion within a Z-expression is similarly straightforward:
\begin{subequations}
\begin{equation}
  \inferrule{ }{
    \performSyn{\hGamma}{\zwsel{\hexp}}{\htau}{\aDel}{\zwsel{\hehole}}{\tehole}
  }
\end{equation}
\begin{equation}
  \inferrule{ }{
    \performAna{\hGamma}{\zwsel{\hexp}}{\htau}{\aDel}{\zwsel{\hehole}}
  }
\end{equation}
\end{subequations}
Unlike the relative movement and construction actions, there is no
``checksum'' theorem for deletion. The rules do not inspect the structure
of the expression in the cursor, so they both match our intuition and will
be correct in any extension of the language without modification.

\subsubsection{Finishing}
The final action we need to consider is $\aFinish$, which finishes the
non-empty hole under the cursor.

If the non-empty hole appears in synthetic position, then it can always be
finished:
\begin{subequations}
  \begin{equation}
  \inferrule{
    \hsyn{\hGamma}{\hexp}{\htau'}
  }{
    \performSyn
      {\hGamma}
      {\zwsel{\hhole{\hexp}}}
      {\tehole}
      {\aFinish}
      {\zwsel{\hexp}}
      {\htau'}
  }
\end{equation}

If the non-empty hole appears in analytic position, then it can only be
finished if the type synthesized for the enveloped expression is consistent
with the type that the hole is being analyzed against. This amounts to
analyzing the enveloped expression against the provided type (by
subsumption):
\begin{equation}\label{r:finishana}
  \inferrule{
    \hana{\hGamma}{\hexp}{\htau}
  }{
    \performAna
      {\hGamma}
      {\zwsel{\hhole{\hexp}}}
      {\htau}
      {\aFinish}
      {\zwsel{\hexp}}
  }
\end{equation}
\end{subequations}
Like deletion, there is no need for a ``checksum'' theorem for the
finishing.


\subsubsection{Zipper Cases}\label{sec:zipper-cases} The rules defined so
far handle the base cases, i.e. the cases where the action has ``reached''
the expression under the cursor. We also need to define the recursive
cases, which propagate the action into the subtree where the cursor
appears, as encoded by the zipper structure. For types, the zipper rules
are straightforward:
\begin{subequations}
\begin{equation}
  \inferrule{
    \performTyp{\ztau}{\alpha}{\ztau'}
  }{
    \performTyp{
      \tarr{\ztau}{\htau}
    }{
      \alpha
    }{
      \tarr{\ztau'}{\htau}
    }
  }
\end{equation}
  \begin{equation}
  \inferrule{
    \performTyp{\ztau}{\alpha}{\ztau'}
  }{
    \performTyp{
      \tarr{\htau}{\ztau}
    }{
      \alpha
    }{
      \tarr{\htau}{\ztau'}
    }
  }
\end{equation}
\end{subequations}
For expressions, the zipper rules essentially follow the structure of the
corresponding rules in the statics.

\begin{subequations}
In particular, when the cursor is in the body of a lambda expression, the
zipper case mirrors Rule (\ref{rule:syn-lam}):
\begin{equation}
\inferrule{
  \arrmatch{\htau}{\tarr{\htau_1}{\htau_2}}\\
  \performAna
    {\hGamma, x : \htau_1}
    {\zexp}
    {\htau_2}
    {\alpha}
    {\zexp'}
}{
  \performAna
    {\hGamma}
    {\hlam{x}{\zexp}}
    {\htau}
    {\alpha}
    {\hlam{x}{\zexp'}}
}
\end{equation}

When the cursor is in the expression position of an ascription, we use the
analytic  judgement, mirroring Rule (\ref{rule:syn-asc}):
\begin{equation}
\inferrule{
  \performAna
    {\hGamma}
    {\zexp}
    {\htau}
    {\alpha}
    {\zexp'}
}{
  \performSyn
    {\hGamma}
    {\zexp : \htau}
    {\htau}
    {\alpha}
    {\zexp' : \htau}
    {\htau}
}
\end{equation}

When the cursor is in the type position of an ascription, we must re-check
the ascribed expression because the cursor erasure might have changed (in
practice, one would optimize this check to only occur if the cursor erasure
did change):
\begin{equation}\label{rule:zipper-asc}
\inferrule{
  \performTyp{\ztau}{\alpha}{\ztau'}\\
  \hana{\hGamma}{\hexp}{\removeSel{\ztau'}}
}{
  \performSyn
    {\hGamma}
    {\hexp : \ztau}
    {\removeSel{\ztau}}
    {\alpha}
    {\hexp : \ztau'}
    {\removeSel{\ztau'}}
}
\end{equation}

When the cursor is in the function position of an application, the rule
mirrors Rule (\ref{rule:syn-ap}):
\begin{equation}
  \inferrule{
    \hsyn{\hGamma}{\removeSel{\zexp}}{\htau_2}\\
    \performSyn
      {\hGamma}
      {\zexp}
      {\htau_2}
      {\alpha}
      {\zexp'}
      {\htau_3}\\\\
    \arrmatch{\htau_3}{\tarr{\htau_4}{\htau_5}}\\
    \hana{\hGamma}{\hexp}{\htau_4}
  }{
    \performSyn
      {\hGamma}
      {\hap{\zexp}{\hexp}}
      {\htau_1}
      {\alpha}
      {\hap{\zexp'}{\hexp}}
      {\htau_5}
  }
\end{equation}

The situation is similar when the cursor is in argument position:
\begin{equation}
  \inferrule{
    \hsyn{\hGamma}{\hexp}{\htau_2}\\
    \arrmatch{\htau_2}{\tarr{\htau_3}{\htau_4}}\\
    \performAna
      {\hGamma}
      {\zexp}
      {\htau_3}
      {\alpha}
      {\zexp'}
  }{
    \performSyn
      {\hGamma}
      {\hap{\hexp}{\zexp}}
      {\htau_1}
      {\alpha}
      {\hap{\hexp}{\zexp'}}
      {\htau_4}
  }
\end{equation}

The rules for the addition operator mirror Rule (\ref{rule:syn-plus}):
\begin{equation}
  \inferrule{
    \performAna
      {\hGamma}
      {\zexp}
      {\tnum}
      {\alpha}
      {\zexp'}
  }{
    \performSyn
      {\hGamma}
      {\hadd{\zexp}{\hexp}}
      {\tnum}
      {\alpha}
      {\hadd{\zexp'}{\hexp}}
      {\tnum}
  }
\end{equation}
\begin{equation}
  \inferrule{
    \performAna
      {\hGamma}
      {\zexp}
      {\tnum}
      {\alpha}
      {\zexp'}
  }{
    \performSyn
      {\hGamma}
      {\hadd{\hexp}{\zexp}}
      {\tnum}
      {\alpha}
      {\hadd{\hexp}{\zexp'}}
      {\tnum}
  }
\end{equation}

Finally, if the cursor is inside a non-empty hole, the relevant zipper rule
mirrors Rule (\ref{rule:syn-ehole}):
\begin{equation}
  \inferrule{
    \hsyn{\hGamma}{\removeSel{\zexp}}{\htau}\\
    \performSyn
      {\hGamma}
      {\zexp}
      {\htau}
      {\alpha}
      {\zexp'}
      {\htau'}\\
  }{
    \performSyn
      {\hGamma}
      {\hhole{\zexp}}
      {\tehole}
      {\alpha}
      {\hhole{\zexp'}}
      {\tehole}
  }
\end{equation}

Theorem \ref{thrm:actsafe} directly checks the correctness of these
rules. Moreover, the zipper rules arise ubiquitously in derivations of edit
steps, so the proofs of the other ``check'' theorems, e.g. Reachability and
Constructability, serve as a check that none of these rules have been
missed.
\end{subequations}
