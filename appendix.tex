% !TEX root = hazelnut-popl17.tex

\section{Hazelnut}
The full collection of rules defining the semantics of Hazelnut are reproduced here in their definitional order for reference.
\subsection{H-Types and H-Expressions}
\subsubsection{Type Compatibility and Incompatibility}

\noindent\fbox{$\tcompat{\htau}{\htau'}$}
\begin{subequations}%%\label{rules:tcompat}
% \begin{equation}%%\label{rule:tcompat-comm}
% \inferrule
% %[TCSym]
% {
%   \tcompat{\htau}{\htau'}
% }{
%   \tcompat{\htau'}{\htau}
% }
% \end{equation}
\begin{equation}
\inferrule{ }{
	\tcompat{\tehole}{\htau}
}
\end{equation}
\begin{equation}%%\label{rule:tcompat-hole}
\inferrule{ }{
  \tcompat{\htau}{\tehole}
}
\end{equation}
\begin{equation}%%\label{rule:tcompat-num}
\inferrule{ }{
  \tcompat{\htau}{\htau}
}
\end{equation}
\begin{equation}%%\label{rule:tcompat-arr}
\inferrule{
  \tcompat{\htau_1}{\htau_1'}\\
  \tcompat{\htau_2}{\htau_2'}
}{
  \tcompat{(\tarr{\htau_1}{\htau_2})}{(\tarr{\htau_1'}{\htau_2'})}
}
\end{equation}
\end{subequations}

\noindent\fbox{$\tincompat{\htau}{\htau'}$}
\begin{subequations}
  % \begin{equation}
  %   \inferrule{
  %     \tincompat{\htau}{\htau'}
  %   }{
  %     \tincompat{\htau'}{\htau}
  %   }
  % \end{equation}
  \begin{equation}
  	\inferrule{ }{
  		\tincompat{\tarr{\htau_1}{\htau_2}}{\tnum}
  	}
  \end{equation}
  \begin{equation}
    \inferrule{ }{
      \tincompat{\tnum}{\tarr{\htau_1}{\htau_2}}
    }
  \end{equation}
  \begin{equation}
    \inferrule{
      \tincompat{\htau_1}{\htau_1'}
    }{
      \tincompat{\tarr{\htau_1}{\htau_2}}{\tarr{\htau_1'}{\htau_2'}}
    }
  \end{equation}
  \begin{equation}
    \inferrule{
      \tincompat{\htau_2}{\htau_2'}
    }{
      \tincompat{\tarr{\htau_1}{\htau_2}}{\tarr{\htau_1'}{\htau_2'}}
    }
  \end{equation}
\end{subequations}

\subsubsection{Function Type Matching}~

% \noindent\fbox{$\tcompat{\htau}{\htau'}$}~~\text{$\tau$ and $\tau'$ are consistent}
% % \begin{subequations}%%\label{rules:tcompat}
% % \begin{equation}%%\label{rule:tcompat-comm}
% % \inferrule
% % %[TCSym]
% % {
% %   \tcompat{\htau}{\htau'}
% % }{
% %   \tcompat{\htau'}{\htau}
% % }
% % \end{equation}
% \begin{equation}
% \inferrule{ }{
%   \tcompat{\tehole}{\htau}
% }
% \end{equation}
% \begin{equation}%%\label{rule:tcompat-hole}
% \inferrule{ }{
%   \tcompat{\htau}{\tehole}
% }

% \inferrule{ }{
%   \tcompat{\htau}{\htau}
% }

% \inferrule{
%   \tcompat{\htau_1}{\htau_1'}\\
%   \tcompat{\htau_2}{\htau_2'}
% }{
%   \tcompat{\tarr{\htau_1}{\htau_2}}{\tarr{\htau_1'}{\htau_2'}}
% }
% \end{mathpar}
% \begin{equation}%%\label{rule:tcompat-num}
% \end{equation}
% \begin{equation}%%\label{rule:tcompat-arr}
% \end{equation}
% \end{subequations}
\noindent
\fbox{$\arrmatch{\htau}{\tarr{\htau_1}{\htau_2}}$}~~\text{$\tau$ has matched arrow type $\tarr{\htau_1}{\htau_2}$}
% \begin{mathpar}
\begin{subequations}
\begin{equation}
\inferrule{ }{
  \arrmatch{\tarr{\htau_1}{\htau_2}}{\tarr{\htau_1}{\htau_2}}
}
\end{equation}
\begin{equation}
\inferrule{ }{
  \arrmatch{\tehole}{\tarr{\tehole}{\tehole}}
}
\end{equation}
\end{subequations}
% \end{mathpar}
% \noindent\fbox{$\tincompat{\htau}{\htau'}$}
% \begin{subequations}
%   % \begin{equation}
%   %   \inferrule{
%   %     \tincompat{\htau}{\htau'}
%   %   }{
%   %     \tincompat{\htau'}{\htau}
%   %   }
%   % \end{equation}
%   \begin{equation}
%     \inferrule{ }{
%       \tincompat{\tarr{\htau_1}{\htau_2}}{\tnum}
%     }
%   \end{equation}
%   \begin{equation}
%     \inferrule{ }{
%       \tincompat{\tnum}{\tarr{\htau_1}{\htau_2}}
%     }
%   \end{equation}
%   \begin{equation}
%     \inferrule{
%       \tincompat{\htau_1}{\htau_1'}
%     }{
%       \tincompat{\tarr{\htau_1}{\htau_2}}{\tarr{\htau_1'}{\htau_2'}}
%     }
%   \end{equation}
%   \begin{equation}
%     \inferrule{
%       \tincompat{\htau_2}{\htau_2'}
%     }{
%       \tincompat{\tarr{\htau_1}{\htau_2}}{\tarr{\htau_1'}{\htau_2'}}
%     }
%   \end{equation}
% \end{subequations}

\subsubsection{Synthesis and Analysis}
The judgements $\hsyn{\hGamma}{\hexp}{\htau}$ and
$\hana{\hGamma}{\hexp}{\htau}$ are defined mutually inductively by Rules
(\ref{Arules:hsyn}) and Rules (\ref{Arules:hana}), respectively.

\fbox{$\hsyn{\hGamma}{\hexp}{\htau}$}~~\text{$\hexp$ synthesizes $\htau$}
\begin{subequations}\label{Arules:hsyn}
\begin{equation}%\label{rule:syn-asc}
\inferrule{
  \hana{\hGamma}{\hexp}{\htau}
}{
  \hsyn{\hGamma}{\hexp : \htau}{\htau}
}
\end{equation}
\begin{equation}%\label{rule:syn-var}
\inferrule{ }{
  \hsyn{\hGamma, x : \htau}{x}{\htau}
}
\end{equation}
\begin{equation}%\label{rule:syn-ap}
\inferrule{
  \hsyn{\hGamma}{\hexp_1}{\htau}\\
  \arrmatch{\htau}{\tarr{\htau_2}{\htau'}}\\
  \hana{\hGamma}{\hexp_2}{\htau_2}
}{
  \hsyn{\hGamma}{\hap{\hexp_1}{\hexp_2}}{\htau'}
}
\end{equation}
\begin{equation}%\label{rule:syn-num}
\inferrule{ }{
  \hsyn{\hGamma}{\hnum{n}}{\tnum}
}
\end{equation}
\begin{equation}%\label{rule:syn-plus}
\inferrule{
  \hana{\hGamma}{\hexp_1}{\tnum}\\
  \hana{\hGamma}{\hexp_2}{\tnum}
}{
  \hsyn{\hGamma}{\hadd{\hexp_1}{\hexp_2}}{\tnum}
}
\end{equation}
\begin{equation}%\label{rule:syn-ehole}
\inferrule{ }{
  \hsyn{\hGamma}{\hehole}{\tehole}
}
\end{equation}
\begin{equation}%\label{rule:syn-hole}
\inferrule{
  \hsyn{\hGamma}{\hexp}{\htau}
}{
  \hsyn{\hGamma}{\hhole{\hexp}}{\tehole}
}
\end{equation}
\end{subequations}
\noindent\fbox{$\hana{\hGamma}{\hexp}{\htau}$}~~\text{$\hexp$ analyzes against $\htau$}
\begin{subequations}\label{Arules:hana}
\begin{equation}%\label{rule:ana-subsume}
\inferrule{
  \hsyn{\hGamma}{\hexp}{\htau'}\\
  \tcompat{\htau}{\htau'}
}{
  \hana{\hGamma}{\hexp}{\htau}
}
\end{equation}
\begin{equation}%\label{rule:syn-lam}
\inferrule{
  \arrmatch{\htau}{\tarr{\htau_1}{\htau_2}}\\
  \hana{\hGamma, x : \htau_1}{\hexp}{\htau_2}
}{
  \hana{\hGamma}{\hlam{x}{\hexp}}{\htau}
}
\end{equation}
\end{subequations}

\subsubsection{Complete H-Types and H-Expressions}
By convention, we use the metavariable $\tau$ rather than $\htau$ for
complete H-types, and $e$ rather than $\hexp$ for complete H-expressions.

\noindent\fbox{$\hcomplete{\tau}$}
\begin{subequations}
\begin{equation}
\inferrule{
  \hcomplete{\tau_1}\\
  \hcomplete{\tau_2}
}{
  \hcomplete{\tarr{\tau_1}{\tau_2}}
}
\end{equation}
\begin{equation}
\inferrule{ }{
  \hcomplete{\tnum}
}
\end{equation}
\end{subequations}

\noindent\fbox{$\hcomplete{e}$}
\begin{subequations}
\begin{equation}
  \inferrule{
    \hcomplete{\hexp}\\
    \hcomplete{\htau}
  }{
    \hcomplete{\hexp : \htau}
  }
\end{equation}
\begin{equation}
  \inferrule{ }{
    \hcomplete{x}
  }
\end{equation}
\begin{equation}
  \inferrule{
    \hcomplete{\hexp}
  }{
    \hcomplete{\hlam{x}{\hexp}}
  }
\end{equation}
\begin{equation}
  \inferrule{
    \hcomplete{\hexp_1}\\
    \hcomplete{\hexp_2}
  }{
    \hcomplete{\hap{\hexp_1}{\hexp_2}}
  }
\end{equation}
\begin{equation}
  \inferrule{ }{\hcomplete{\hnum{n}}}
\end{equation}
\begin{equation}
  \inferrule{
    \hcomplete{\hexp_1}\\
    \hcomplete{\hexp_2}
  }{
    \hcomplete{\hadd{\hexp_1}{\hexp_2}}
  }
\end{equation}
\end{subequations}

\subsection{Z-Types and Z-Expressions}
\subsubsection{Type Focus Erasure}
\noindent\fbox{$\removeSel{\ztau}=\htau$} is a metafunction defined as follows:
\begin{subequations}
\begin{align}
%\removeSel{(\zlsel{\htau})} & = \htau\\
\removeSel{(\zwsel{\htau})} & = \htau\\
%\removeSel{(\zrsel{\htau})} & = \htau\\
\removeSel{(\tarr{\ztau}{\htau})} & = \tarr{\removeSel{\ztau}}{\htau}\\
\removeSel{(\tarr{\htau}{\ztau})} & = \tarr{\htau}{\removeSel{\ztau}}
\end{align}
\end{subequations}

\subsubsection{Expression Focus Erasure}
\noindent\fbox{$\removeSel{\zexp}=\hexp$} is a metafunction defined as follows:
\begin{subequations}
\begin{align}
%\removeSel{(\zlsel{\hexp})} & = \hexp\\
\removeSel{(\zwsel{\hexp})} & = \hexp\\
%\removeSel{(\zrsel{\hexp})} & = \hexp\\
\removeSel{(\zexp : \htau)} & = \removeSel{\zexp} : \htau\\
\removeSel{(\hexp : \ztau)} & = \hexp : \removeSel{\ztau}\\
\removeSel{(\hlam{x}{\zexp})} & = \hlam{x}{\removeSel{\zexp}}\\
\removeSel{(\hap{\zexp}{\hexp})} & = \hap{\removeSel{\zexp}}{\hexp}\\
\removeSel{(\hap{\hexp}{\zexp})} & = \hap{\hexp}{\removeSel{\zexp}}\\
\removeSel{(\hadd{\zexp}{\hexp})} & = \hadd{\removeSel{\zexp}}{\hexp}\\
\removeSel{(\hadd{\hexp}{\zexp})} & = \hadd{\hexp}{\removeSel{\zexp}}\\
\removeSel{\hhole{\zexp}} &= \hhole{\removeSel{\zexp}}
\end{align}
\end{subequations}
\subsection{Action Model}
\subsubsection{Type Actions}
\noindent\fbox{$\performTyp{\ztau}{\alpha}{\ztau'}$}
\paragraph{Type Movement}
\begin{subequations}
\begin{equation}
  \inferrule{ }{
    \performTyp{
      \zwsel{\tarr{\htau_1}{\htau_2}}
    }{
      \aMove{\dChild}
    }{
      \tarr{\zwsel{\htau_1}}{\htau_2}
    }
  }
\end{equation}
\begin{equation}
  \inferrule{ }{
    \performTyp{
      \tarr{\zwsel{\htau_1}}{\htau_2}
    }{
      \aMove{\dParent}
    }{
      \zwsel{\tarr{\htau_1}{\htau_2}}
    }
  }
\end{equation}
\begin{equation}
  \inferrule{ }{
    \performTyp{
      \tarr{{\htau_1}}{\zwsel{\htau_2}}
    }{
      \aMove{\dParent}
    }{
      \zwsel{\tarr{\htau_1}{\htau_2}}
    }
  }
\end{equation}
\begin{equation}
  \inferrule{ }{
    \performTyp{
      \tarr{\zwsel{\htau_1}}{{\htau_2}}
    }{
      \aMove{\dNext}
    }{
      {\tarr{\htau_1}{\zwsel{\htau_2}}}
    }
  }
\end{equation}
% \begin{equation}
%   \inferrule{ }{
%     \performTyp{
%       \tarr{{\htau_1}}{\zwsel{\htau_2}}
%     }{
%       \aMove{\dPrev}
%     }{
%       {\tarr{\zwsel{\htau_1}}{{\htau_2}}}
%     }
%   }
% \end{equation}

\paragraph{Type Deletion}
\begin{equation}
  \inferrule{ }{
    \performTyp{
      \zwsel{\htau}
    }{
      \aDel
    }{
      \zwsel{\tehole}
    }
  }
\end{equation}

\paragraph{Type Construction}
\begin{equation}
    %%\label{r:contarr}
  \inferrule{ }{
    \performTyp{
      \zwsel{\htau}
    }{
      \aConstruct{\farr}
    }{
      \tarr{\htau}{\zwsel{\tehole}}
    }
  }
\end{equation}

  \begin{equation}
    %%\label{r:contnum}
  \inferrule{ }{
    \performTyp{
      \zwsel{\tehole}
    }{
      \aConstruct{\fnum}
    }{
      \zwsel{\tnum}
    }
  }
\end{equation}


\paragraph{Zipper Cases}
  \begin{equation}
    %%\label{r:contarrL}
  \inferrule{
    \performTyp{\ztau}{\alpha}{\ztau'}
  }{
    \performTyp{
      \tarr{\ztau}{\htau}
    }{
      \alpha
    }{
      \tarr{\ztau'}{\htau}
    }
  }
\end{equation}
  \begin{equation}
    %%\label{r:contarrR}
  \inferrule{
    \performTyp{\ztau}{\alpha}{\ztau'}
  }{
    \performTyp{
      \tarr{\htau}{\ztau}
    }{
      \alpha
    }{
      \tarr{\htau}{\ztau'}
    }
  }
\end{equation}
\end{subequations}

\subsubsection{Expression Movement Actions} 
\noindent\fbox{$\performMove{\zexp}{\aMove{\delta}}{\zexp'}$}

\begin{subequations}
\paragraph{Ascription}

\begin{equation}
  \inferrule{ }{
    \performMove{
      \zwsel{\hexp : \htau}
    }{
      \aMove{\dChild}
    }{
      \zwsel{\hexp} : \htau
    }
  }
\end{equation}
\begin{equation}
  \inferrule{ }{
    \performMove{
      \zwsel{\hexp} : \htau
    }{
      \aMove{\dParent}
    }{
      \zwsel{\hexp : \htau}
    }
  }
\end{equation}
\begin{equation}
  \inferrule{ }{
    \performMove{
      \hexp : \zwsel{\htau}
    }{
      \aMove{\dParent}
    }{
      \zwsel{\hexp : \htau}
    }
  }
\end{equation}
\begin{equation}
  \inferrule{ }{
    \performMove{
      \zwsel{\hexp} : \htau
    }{
      \aMove{\dNext}
    }{
      \hexp : \zwsel{\htau}
    }
  }
\end{equation}
% \begin{equation}
%   \inferrule{ }{
%     \performMove{
%       \hexp : \zwsel{\htau}
%     }{
%       \aMove{\dPrev}
%     }{
%       \zwsel{\hexp} : \htau
%     }
%   }
% \end{equation}
% \begin{equation}
% \inferrule{
%   \performMove{
%     \zexp
%   }{
%     \aMove{\delta}
%   }{
%     \zexp'
%   }
% }{
%   \performMove{
%     \zexp : \htau
%   }{
%     \aMove{\delta}
%   }{
%     \zexp' : \htau
%   }
% }
% \end{equation}
% \begin{equation}
%   \inferrule{
%     \performMove{
%       \ztau
%     }{
%       \aMove{\delta}
%     }{
%       \ztau'
%     }
%   }{
%     \performMove{
%       \hexp : \ztau
%     }{
%       \aMove{\delta}
%     }{
%       \hexp : \ztau'
%     }
%   }
% \end{equation}

\paragraph{Lambda}
\begin{equation}%\label{r:movefirstchild-lam}
\inferrule{ }{
  \performMove{
    \zwsel{\hlam{x}{\hexp}}
  }{
    \aMove{\dChild}
  }{
    \hlam{x}{\zwsel{\hexp}}
  }
}
\end{equation}
\begin{equation}
  \inferrule{ }{
    \performMove{
      \hlam{x}{\zwsel{\hexp}}
    }{
      \aMove{\dParent}
    }{
      \zwsel{\hlam{x}{\hexp}}
    }
  }
\end{equation}
\paragraph{Application}
\begin{equation}
  \inferrule{ }{
    \performMove{
      \zwsel{\hap{\hexp_1}{\hexp_2}}
    }{
      \aMove{\dChild}
    }{
      \hap{\zwsel{\hexp_1}}{\hexp_2}
    }
  }
\end{equation}
\begin{equation}
  \inferrule{ }{
    \performMove{
      \hap{\zwsel{\hexp_1}}{\hexp_2}
    }{
      \aMove{\dParent}
    }{
      \zwsel{\hap{\hexp_1}{\hexp_2}}
    }
  }
\end{equation}
\begin{equation}%\label{r:moveparent-ap2}
  \inferrule{ }{
    \performMove{
      \hap{{\hexp_1}}{\zwsel{\hexp_2}}
    }{
      \aMove{\dParent}
    }{
      \zwsel{\hap{\hexp_1}{\hexp_2}}
    }
  }
\end{equation}
\begin{equation}
  \inferrule{ }{
    \performMove{
      \hap{\zwsel{\hexp_1}}{\hexp_2}
    }{
      \aMove{\dNext}
    }{
      \hap{\hexp_1}{\zwsel{\hexp_2}}
    }
  }
\end{equation}
% \begin{equation}
%   \inferrule{ }{
%     \performMove{
%       \hap{\hexp_1}{\zwsel{\hexp_2}}
%     }{
%       \aMove{\dPrev}
%     }{
%       \hap{\zwsel{\hexp_1}}{\hexp_2}
%     }
%   }
% \end{equation}

\paragraph{Plus}
\begin{equation}
  \inferrule{ }{
    \performMove{
      \zwsel{\hadd{\hexp_1}{\hexp_2}}
    }{
      \aMove{\dChild}
    }{
      \hadd{\zwsel{\hexp_1}}{\hexp_2}
    }
  }
\end{equation}
\begin{equation}
  \inferrule{ }{
    \performMove{
      \hadd{\zwsel{\hexp_1}}{\hexp_2}
    }{
      \aMove{\dParent}
    }{
      \zwsel{\hadd{\hexp_1}{\hexp_2}}
    }
  }
\end{equation}
\begin{equation}
  \inferrule{ }{
    \performMove{
      \hadd{{\hexp_1}}{\zwsel{\hexp_2}}
    }{
      \aMove{\dParent}
    }{
      \zwsel{\hadd{\hexp_1}{\hexp_2}}
    }
  }
\end{equation}
\begin{equation}
  \inferrule{ }{
    \performMove{
      \hadd{\zwsel{\hexp_1}}{\hexp_2}
    }{
      \aMove{\dNext}
    }{
      \hadd{\hexp_1}{\zwsel{\hexp_2}}
    }
  }
\end{equation}
% \begin{equation}
%   \inferrule{ }{
%     \performMove{
%       \hadd{\hexp_1}{\zwsel{\hexp_2}}
%     }{
%       \aMove{\dPrev}
%     }{
%       \hadd{\zwsel{\hexp_1}}{\hexp_2}
%     }
%   }
% \end{equation}

\paragraph{Non-Empty Hole}
\begin{equation}
\inferrule{ }{
  \performMove{
    \zwsel{\hhole{\hexp}}
  }{
    \aMove{\dChild}
  }{
    \hhole{\zwsel{\hexp}}
  }
}
\end{equation}
\begin{equation}%\label{r:moveparent-hole}
  \inferrule{ }{
    \performMove{
      \hhole{\zwsel{\hexp}}
    }{
      \aMove{\dParent}
    }{
      \zwsel{\hhole{\hexp}}
    }
  }
\end{equation}

\end{subequations}
\subsubsection{Synthetic and Analytic Expression Actions}
The synthetic and analytic expression action performance judgements are defined mutually inductively by Rules (\ref{Arules:performSyn}) and Rules (\ref{Arules:performAna}), respectively.


\noindent\fbox{$\performSyn{\hGamma}{\zexp}{\htau}{\alpha}{\zexp'}{\htau'}$}

\begin{subequations}\label{Arules:performSyn}
\paragraph{Movement}
\begin{equation}
\inferrule{
  \performMove{\zexp}{\aMove{\delta}}{\zexp'}
}{
  \performSyn{\hGamma}{\zexp}{\htau}{\aMove{\delta}}{\zexp'}{\htau}
}
\end{equation}

\paragraph{Deletion}
\begin{equation}
  \inferrule{ }{
    \performSyn{\hGamma}{\zwsel{\hexp}}{\htau}{\aDel}{\zwsel{\hehole}}{\tehole}
  }
\end{equation}

\paragraph{Construction}
\begin{equation}
  \inferrule{ }{
    \performSyn{\hGamma}{\zwsel{\hexp}}{\htau}{\aConstruct{\fasc}}{\hexp : \zwsel{\htau}}{\htau}
  }
\end{equation}

\begin{equation}
  \inferrule{ }{
    \performSyn{\hGamma, x : \htau}{\zwsel{\hehole}}{\tehole}{\aConstruct{\fvar{x}}}{\zwsel{x}}{\htau}
  }
\end{equation}

\begin{equation}
  \inferrule{ }{
    \performSyn
      {\hGamma}
      {\zwsel{\hehole}}
      {\tehole}
      {\aConstruct{\flam{x}}}
      {\hlam{x}{\hehole} : \tarr{\zwsel{\tehole}}{\tehole}}
      {\tarr{\tehole}{\tehole}}
  }
\end{equation}

\begin{equation}
  %\label{r:coneapfn}
  \inferrule{
    \arrmatch{\htau}{\tarr{\htau_1}{\htau_2}}
  }{
    \performSyn
      {\hGamma}
      {\zwsel{\hexp}}
      {\htau}
      {\aConstruct{\fap}}
      {\hap{\hexp}{\zwsel{\hehole}}}
      {\htau_2}
  }
\end{equation}

% \begin{equation}
%   \inferrule{ }{
%     \performSyn
%       {\hGamma}
%       {\zwsel{\hexp}}
%       {\tehole}
%       {\aConstruct{\fap}}
%       {\hap{\hexp}{\zwsel{\hehole}}}
%       {\tehole}
%   }
% \end{equation}

\begin{equation}
  \inferrule{
    \tincompat{\htau}{\tarr{\tehole}{\tehole}}
  }{
    \performSyn
      {\hGamma}
      {\zwsel{\hexp}}
      {\htau}
      {\aConstruct{\fap}}
      {\hap{\hhole{\hexp}}{\zwsel{\hehole}}}
      {\tehole}
  }
\end{equation}

\begin{equation}
  \inferrule{ }{
    \performSyn
      {\hGamma}
      {\zwsel{\hexp}}
      {\htau}
      {\aConstruct{\farg}}
      {\hap{\zwsel{\hehole}}{\hexp}}
      {\tehole}
  }
\end{equation}

\begin{equation}
  \inferrule{ }{
    \performSyn
      {\hGamma}
      {\zwsel{\hehole}}
      {\tehole}
      {\aConstruct{\fnumlit{n}}}
      {\zwsel{\hnum{n}}}
      {\tnum}
  }
\end{equation}

\begin{equation}
  \inferrule{
    \tcompat{\htau}{\tnum}
  }{
    \performSyn
      {\hGamma}
      {\zwsel{\hexp}}
      {\htau}
      {\aConstruct{\fplus}}
      {\hadd{\hexp}{\zwsel{\hehole}}}
      {\tnum}
  }
\end{equation}

\begin{equation}
  \inferrule{
    \tincompat{\htau}{\tnum}
  }{
    \performSyn
      {\hGamma}
      {\zwsel{\hexp}}
      {\htau}
      {\aConstruct{\fplus}}
      {\hadd{\hhole{\hexp}}{\zwsel{\hehole}}}
      {\tnum}
  }
\end{equation}

\begin{equation}
\inferrule{ }{
  \performSyn
    {\hGamma}
    {\zwsel{\hexp}}
    {\htau}
    {\aConstruct{\fnehole}}
    {\hhole{\zwsel{\hexp}}}
    {\tehole}
}
\end{equation}
\paragraph{Finishing}
  \begin{equation}
    %% %\label{r:finishana} %% TODO; get labels right
  \inferrule{
    \hsyn{\hGamma}{\hexp}{\htau'}
  }{
    \performSyn
      {\hGamma}
      {\zwsel{\hhole{\hexp}}}
      {\tehole}
      {\aFinish}
      {\zwsel{\hexp}}
      {\htau'}
  }
\end{equation}


\paragraph{Zipper Cases}
\begin{equation}
\inferrule{
  \performAna
    {\hGamma}
    {\zexp}
    {\htau}
    {\alpha}
    {\zexp'}
}{
  \performSyn
    {\hGamma}
    {\zexp : \htau}
    {\htau}
    {\alpha}
    {\zexp' : \htau}
    {\htau}
}
\end{equation}
\begin{equation}
\inferrule{
  \performTyp{\ztau}{\alpha}{\ztau'}\\
  \hana{\hGamma}{\hexp}{\removeSel{\ztau'}}
}{
  \performSyn
    {\hGamma}
    {\hexp : \ztau}
    {\removeSel{\ztau}}
    {\alpha}
    {\hexp : \ztau'}
    {\removeSel{\ztau'}}
}
\end{equation}
\begin{equation}
  \inferrule{
    \hsyn{\hGamma}{\removeSel{\zexp}}{\htau_2}\\
    \performSyn
      {\hGamma}
      {\zexp}
      {\htau_2}
      {\alpha}
      {\zexp'}
      {\htau_3}\\\\
    \arrmatch{\htau_3}{\tarr{\htau_4}{\htau_5}}\\
    \hana{\hGamma}{\hexp}{\htau_4}
  }{
    \performSyn
      {\hGamma}
      {\hap{\zexp}{\hexp}}
      {\htau_1}
      {\alpha}
      {\hap{\zexp'}{\hexp}}
      {\htau_5}
  }
\end{equation}
\begin{equation}
  \inferrule{
    \hsyn{\hGamma}{\hexp}{\htau_2}\\
    \arrmatch{\htau_2}{\tarr{\htau_3}{\htau_4}}\\
    \performAna
      {\hGamma}
      {\zexp}
      {\htau_3}
      {\alpha}
      {\zexp'}
  }{
    \performSyn
      {\hGamma}
      {\hap{\hexp}{\zexp}}
      {\htau_1}
      {\alpha}
      {\hap{\hexp}{\zexp'}}
      {\htau_4}
  }
\end{equation}

\begin{equation}
  \inferrule{
    \performAna
      {\hGamma}
      {\zexp}
      {\tnum}
      {\alpha}
      {\zexp'}
  }{
    \performSyn
      {\hGamma}
      {\hadd{\zexp}{\hexp}}
      {\tnum}
      {\alpha}
      {\hadd{\zexp'}{\hexp}}
      {\tnum}
  }
\end{equation}

\begin{equation}
  \inferrule{
    \performAna
      {\hGamma}
      {\zexp}
      {\tnum}
      {\alpha}
      {\zexp'}
  }{
    \performSyn
      {\hGamma}
      {\hadd{\hexp}{\zexp}}
      {\tnum}
      {\alpha}
      {\hadd{\hexp}{\zexp'}}
      {\tnum}
  }
\end{equation}

\begin{equation}
  \inferrule{
    \hsyn{\hGamma}{\removeSel{\zexp}}{\htau}\\
    \performSyn
      {\hGamma}
      {\zexp}
      {\htau}
      {\alpha}
      {\zexp'}
      {\htau'}
  }{
    \performSyn
      {\hGamma}
      {\hhole{\zexp}}
      {\tehole}
      {\alpha}
      {\hhole{\zexp'}}
      {\tehole}
  }
\end{equation}
% \begin{equation}
%   \inferrule{
%     \hsyn{\hGamma}{\removeSel{\zexp}}{\htau}\\
%     \performSyn
%       {\hGamma}
%       {\zexp}
%       {\htau}
%       {\alpha}
%       {\zwsel{\hehole}}
%       {\tehole}\\
%   }{
%     \performSyn
%       {\hGamma}
%       {\hhole{\zexp}}
%       {\tehole}
%       {\alpha}
%       {\zwsel{\hehole}}
%       {\tehole}
%   }
% \end{equation}
\end{subequations}

\noindent\fbox{$\performAna{\hGamma}{\zexp}{\htau}{\alpha}{\zexp'}$}
\begin{subequations}\label{Arules:performAna}
\paragraph{Subsumption}
\begin{equation}
  \inferrule{
    \hsyn{\hGamma}{\removeSel{\zexp}}{\htau'}\\
    \performSyn{\hGamma}{\zexp}{\htau'}{\alpha}{\zexp'}{\htau''}\\
    \tcompat{\htau}{\htau''}%\\\\
    % \alpha \neq \aConstruct{\fasc}\\
    % \alpha \neq \aConstruct{\flam{x}}
  }{
    \performAna{\hGamma}{\zexp}{\htau}{\alpha}{\zexp'}
  }
\end{equation}

\paragraph{Movement}
\begin{equation}
  \inferrule{
  \performMove{\zexp}{\aMove{\delta}}{\zexp'}
}{
  \performAna{\hGamma}{\zexp}{\htau}{\aMove{\delta}}{\zexp'}
}
\end{equation}

\paragraph{Deletion}
\begin{equation}
  \inferrule{ }{
    \performAna{\hGamma}{\zwsel{\hexp}}{\htau}{\aDel}{\zwsel{\hehole}}
  }
\end{equation}

\paragraph{Construction}
\begin{equation}
  \inferrule{ }{
    \performAna{\hGamma}{\zwsel{\hexp}}{\htau}{\aConstruct{\fasc}}{\hexp : \zwsel{\htau}}
  }
\end{equation}

\begin{equation}
  \inferrule{
    \tincompat{\htau}{\htau'}
  }{
    \performAna{\hGamma, x : \htau'}{\zwsel{\hehole}}{\htau}{\aConstruct{\fvar{x}}}{\hhole{\zwsel{x}}}
  }
\end{equation}

\begin{equation}%\label{rule:performAna-lam-1}
  \inferrule{
    \arrmatch{\htau}{\tarr{\htau_1}{\htau_2}}
  }{
    \performAna
      {\hGamma}
      {\zwsel{\hehole}}
      {\htau}
      {\aConstruct{\flam{x}}}
      {\hlam{x}{\zwsel{\hehole}}}
  }
\end{equation}

\begin{equation}
  \inferrule{
    \tincompat{\htau}{\tarr{\tehole}{\tehole}}
  }{
    \performAna
      {\hGamma}
      {\zwsel{\hehole}}
      {\htau}
      {\aConstruct{\flam{x}}}
      {\hhole{
        \hlam{x}{\hehole} : \tarr{\zwsel{\tehole}}{\tehole}
      }}
  }
\end{equation}
\begin{equation}
  \inferrule{
    \tincompat{\htau}{\tnum}
  }{
    \performAna
      {\hGamma}
      {\zwsel{\hehole}}
      {\htau}
      {\aConstruct{\fnumlit{n}}}
      {\hhole{\zwsel{\hnum{n}}}}
  }
\end{equation}
\paragraph{Finishing}
\begin{equation}
  \inferrule{
    \hana{\hGamma}{\hexp}{\htau}
  }{
    \performAna
      {\hGamma}
      {\zwsel{\hhole{\hexp}}}
      {\htau}
      {\aFinish}
      {\zwsel{\hexp}}
  }
\end{equation}

\paragraph{Zipper Cases}
\begin{equation}
\inferrule{
  \arrmatch{\htau}{\tarr{\htau_1}{\htau_2}}\\
  \performAna
    {\hGamma, x : \htau_1}
    {\zexp}
    {\htau_2}
    {\alpha}
    {\zexp'}
}{
  \performAna
    {\hGamma}
    {\hlam{x}{\zexp}}
    {\htau}
    {\alpha}
    {\hlam{x}{\zexp'}}
}
\end{equation}

\end{subequations}
\subsubsection{Iterated Action Judgements} ~

\noindent $\mathsf{ActionList}$~~$\bar{\alpha} ::= \cdot ~\vert~ \alpha; \bar{\alpha}$\vspace{4px}\\
\fbox{$\performTypI{\ztau}{\bar{\alpha}}{\ztau'}$}
\begin{subequations}
\begin{equation}
\inferrule{ }{
    \performTypI{\ztau}{\cdot}{\ztau}
}
\end{equation}
\begin{equation}
\inferrule{
  \performTyp{\ztau}{\alpha}{\ztau'}\\
  \performTypI{\ztau'}{\bar{\alpha}}{\ztau''}
}{
  \performTypI{\ztau}{\alpha; \bar{\alpha}}{\ztau''}
}
\end{equation}
\end{subequations}
\begin{subequations}
\fbox{$\performSynI{\hGamma}{\zexp}{\htau}{\bar{\alpha}}{\zexp'}{\htau'}$}
\begin{equation}
\inferrule{ }{
  \performSynI{\hGamma}{\zexp}{\htau}{\cdot}{\zexp}{\htau}
}
\end{equation}
\begin{equation}
\inferrule{
  \performSyn{\hGamma}{\zexp}{\htau}{\alpha}{\zexp'}{\htau'}\\
  \performSynI{\hGamma}{\zexp'}{\htau'}{\bar{\alpha}}{\zexp''}{\htau''}
}{
  \performSynI{\hGamma}{\zexp}{\htau}{\alpha; \bar{\alpha}}{\zexp''}{\htau''}
}
\end{equation}
\end{subequations}
\begin{subequations}
\fbox{$\performAnaI{\hGamma}{\zexp}{\htau}{\bar{\alpha}}{\zexp'}$}
\begin{equation}
\inferrule{ }{
  \performAnaI{\hGamma}{\zexp}{\htau}{\cdot}{\zexp}
}
\end{equation}
\begin{equation}
\inferrule{
  \performAna{\hGamma}{\zexp}{\htau}{\alpha}{\zexp'}\\
  \performAnaI{\hGamma}{\zexp'}{\htau}{\bar\alpha}{\zexp''}
}{
  \performAnaI{\hGamma}{\zexp}{\htau}{\alpha; \bar\alpha}{\zexp''}
}
\end{equation}
\end{subequations}
\noindent \fbox{$\bar\alpha~\mathsf{movements}$}
\begin{subequations}
\begin{equation}
\inferrule{ }{
	\cdot~\mathsf{movements}
}
\end{equation}

\begin{equation}
\inferrule{
	\bar\alpha~\mathsf{movements}
}{
	\aMove{\delta}; \bar\alpha~\mathsf{movements}
}
\end{equation}
\end{subequations}
