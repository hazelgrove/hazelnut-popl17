% !TEX root = hazelnut-popl17.tex
\subsection{Implementation Concepts}
Central to any implementation of Hazelnut is a stream of edit states whose cursor erasures synthesize types according to the synthetic action judgement under an empty context, $\performSyn{\emptyset}{\zexp}{\htau}{\alpha}{\zexp'}{\htau'}$. The middle row of Figure \ref{fig:impl-overview} diagrams this stream of edit states. By Theorem \ref{thrm:actsafe}, the editor does not need to typecheck each edit state (though in some of the rules, e.g. Rule (\ref{rule:zipper-asc}) which handles the situation where the type in a type ascription changes, portions of the program do need to be typechecked.) For example, the reader is encouraged to re-examine the examples in Figure \ref{fig:first-example} and \ref{fig:second-example} -- the cursor erasure of each edit state synthesizes a type.

The programmer examines a view generated from each edit state and produces actions in some implementation-defined manner (e.g. using a keyboard, mouse, touchscreen, voice interface, or neural implant.) Each new action causes a new abstract edit state to arise according to an implementation of the action semantics. This then causes a new to arise. This is a simple event-based functional reactive programming model \cite{Wan:2000:FRP:349299.349331}. 

If an action is not well-defined according to the action semantics, the implementation must reject it. In fact, an implementation is encouraged to present an ``action palette'' that either hides or visibly disables actions that are not well-defined (see below.)



The key question that must be answered for any implementation strategy is: how do we model a stream of actions from a user? Let us assume that these actions are chosen (using some input device, preferably, a keyboard) from some ``palette'' that never presents the user with actions that are not semantically well-defined, according to the action semantics defined earlier.
%In a traditional editor, the input from the user is a stream of characters, and there are no guarantees that at any point that the program is syntactically well-formed, so the designer leaves editing as an .
%In contrast, in a structure editor, the input from the user is a stream of operations.
As such, each new action will ``atomically'' generate a new Z-expression. 
This insight leads us to conclude that a natural way to implement this editor would be using event-based Functional Reactive Programming~\cite{Wan:2000:FRP:349299.349331} (FRP).
Figure~\ref{fig:FRP} illustrates the concept of an FRP-based implementation of a  structure editor organized like Hazelnut.
The input from the user is a stream of actions.  Each action results in a change to the underlying abstract model (i.e., a new Z-expression is created after each action.)
Each model change results in an updated \emph{view} which is then presented to the user.  The user can then consider this new view when they choose a new action as input.

\subsection{HZ}
We explore the concepts presented in the paper in HZ, our implementation of Hazelnut.
In order to reach a wide audience, we decided to implement HZ in the web browser.
In order to take advantage of all the benefits of FRP, we chose to implement HZ using OCaml\footnote{https://ocaml.org/}, the \texttt{js\_of\_ocaml} compiler\footnote{http://ocsigen.org/js\_of\_ocaml/} and the OCaml React library\footnote{http://erratique.ch/software/react}.

At the time of the writing of this paper, our implementation of HZ includes encodings of Z-expression as presented in this paper.
We consider this ZExp to be our model. 
HZ renders the model as a string embedded in HTML.
Currently we support only the delete action.  Other actions are currently under development. We anticipate having a substantially more functional implementation by the time this work is presented (our focus thusfar has been on the metatheory.) 
The work-in-progress code as well as directions for how to compile and run it can be found here: \url{https://github.com/hazelgrove/impl-tfp16}.

\begin{figure}
\centering
\includegraphics[width=0.90\columnwidth]{impl-overview2}
\caption{...}
\label{fig:impl-overview}
\end{figure}
