% !TEX root = hazelnut-popl17.tex
% So far, we have given an overview of the most important judgements and
% rules in the semantics of Hazelnut, and stated the critical metatheorem,
% Sensibility, and several auxiliary ``checks''. In a few cases, we have
% informally sketched out why these metatheorems will hold. 

In order to
formally establish that our design meets our stated objectives, we have
mechanized the semantics and metatheory of Hazelnut as described above using the Agda proof
assistant \cite{norell:thesis} (also see the Agda Wiki, hosted
at \url{http://wiki.portal.chalmers.se/agda/}.) This development is available in the supplemental material. The mechanization also includes the  extension to Hazelnut described in Sec. \ref{sec:extending}.

The documentation includes a more detailed discussion of the technical
representation decisions that we made. The main idea is standard: we encode
each judgement as a dependent type. The rules defining the judgements
become the constructors of this type, and derivations are terms of these
type. This is a rich setting that allows proofs to take advantage of
pattern matching on the shape of derivations, closely matching standard
on-paper proofs. No proof automation was used, because the proof structure
itself is likely to be interesting to researchers who plan to build upon
our work.

We adopt Barendregt's convention for bound
variables \cite{urban}. Hazelnut's semantics does not need substitution, so
we do not need to adopt more sophisticated encodings
(e.g. \cite{lh09unibind,Pouillard11}.)
