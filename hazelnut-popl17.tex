%\documentclass[12pt]{article}
\documentclass[preprint,9pt]{sigplanconf}
% The following \documentclass options may be useful:
%
% 10pt          To set in 10-point type instead of 9-point.
% 11pt          To set in 11-point type instead of 9-point.
% authoryear    To obtain author/year citation style instead of numeric.
\usepackage{amsmath}
\usepackage{amssymb}
\usepackage{amsthm}
\usepackage{ stmaryrd }
\usepackage{mathpartir}
\usepackage{stmaryrd}
\usepackage{wasysym}
\usepackage{extarrows}
\usepackage[usenames,dvipsnames,svgnames,table]{xcolor}
\usepackage{mathpazo}
\usepackage{colortab}
\usepackage{url}
\usepackage{todonotes}
\sloppy
\def \TirNameStyle #1{\small\rulename{#1}}
\renewcommand{\MathparLineskip}{\lineskiplimit=.3\baselineskip\lineskip=.3\baselineskip plus .2\baselineskip}

\newtheorem{theorem}{Theorem}
\newtheorem{lemma}{Lemma}
\newtheorem{corollary}{Corollary}
\newtheorem{definition}{Definition}
\newenvironment{proof-sketch}{\noindent{\emph{Proof Sketch.}}}{\qed}
\makeatletter

\renewcommand\topfraction{0.85}
\renewcommand\bottomfraction{0.85}
\renewcommand\textfraction{0.1}
\renewcommand\floatpagefraction{0.85}

\AtBeginDocument{%
 \abovedisplayskip=3pt
 \abovedisplayshortskip=0pt
 \belowdisplayskip=2pt
 \belowdisplayshortskip=0pt
}

\setlength{\floatsep}{10pt}
\setlength{\textfloatsep}{12pt}

\usepackage[compact]{titlesec}
\titlespacing*{\section}{0pt}{3pt}{3pt}
\titlespacing*{\subsection}{0pt}{4pt}{3pt}
\titlespacing*{\subsubsection}{0pt}{4pt}{3pt}
\titlespacing*{\paragraph}{0pt}{3pt}{5pt}
\setlength{\skip\footins}{3pt}

\usepackage[colorlinks=true,allcolors=blue,breaklinks,draft=false]{hyperref}   % hyperlinks, including DOIs and URLs in bibliography
% known bug: http://tex.stackexchange.com/questions/1522/pdfendlink-ended-up-in-different-nesting-level-than-pdfstartlink

\usepackage{enumitem}

% !TEX root = editor-tfp16.tex

% HTyp and HExp
\newcommand{\hcomplete}[1]{#1~\mathsf{complete}}

% HTyp
\newcommand{\htau}{\dot{\tau}}
\newcommand{\tarr}[2]{#1 \rightarrow #2}
\newcommand{\tnum}{\texttt{num}}
\newcommand{\tehole}{\llparenthesis\rrparenthesis}

\newcommand{\tcompat}[2]{#1 \sim #2}
\newcommand{\tincompat}[2]{#1 \nsim #2}

% HExp
\newcommand{\hexp}{\dot{e}}
\newcommand{\hlam}[2]{\lambda #1.#2}
\newcommand{\hap}[2]{#1(#2)}
\newcommand{\hapP}[2]{(#1)~(#2)} % Extra paren around function term
\newcommand{\hnum}[1]{\underline{#1}}
\newcommand{\hadd}[2]{#1 + #2}
\newcommand{\hehole}{\llparenthesis\rrparenthesis}
\newcommand{\hhole}[1]{\llparenthesis#1\rrparenthesis}

\newcommand{\hGamma}{\dot{\Gamma}}
\newcommand{\domof}[1]{\text{dom}(#1)}
\newcommand{\hsyn}[3]{#1 \vdash #2 \Rightarrow #3}
\newcommand{\hana}[3]{#1 \vdash #2 \Leftarrow #3}

% ZTyp and ZExp
\newcommand{\zlsel}[1]{{\bowtie}{#1}}
\newcommand{\zrsel}[1]{{#1}{\bowtie}}
\newcommand{\zwsel}[1]{{\triangleright}{#1}{\triangleleft}}

\newcommand{\removeSel}[1]{#1^{\diamond}}

% ZTyp
\newcommand{\ztau}{\hat{\tau}}

% ZExp
\newcommand{\zexp}{\hat{e}}

% Direction
\newcommand{\dParent}{\mathtt{parent}}
\newcommand{\dChild}{\mathtt{firstChild}}
\newcommand{\dNext}{\mathtt{nextSib}}
\newcommand{\dPrev}{\mathtt{prevSib}}

% Action
\newcommand{\aMove}[1]{\mathtt{move}~#1}
	\newcommand{\zrightmost}[1]{\mathsf{rightmost}(#1)}
	\newcommand{\zleftmost}[1]{\mathsf{leftmost}(#1)}
\newcommand{\aSelect}[1]{\mathtt{sel}~#1}
\newcommand{\aDel}{\mathtt{del}}
\newcommand{\aReplace}[1]{\mathtt{replace}~#1}
\newcommand{\aConstruct}[1]{\mathtt{construct}~#1}
\newcommand{\aFinish}{\mathtt{finish}}

\newcommand{\performAna}[5]{#1 \vdash #2 \xlongrightarrow{#4} #5 \Leftarrow #3}
\newcommand{\performSyn}[6]{#1 \vdash #2 \Rightarrow #3 \xlongrightarrow{#4} #5 \Rightarrow #6}
\newcommand{\performTyp}[3]{#1 \xlongrightarrow{#2} #3}

\newcommand{\performMove}[3]{#1 \xlongrightarrow{#2} #3}
\newcommand{\performDel}[2]{#1 \xlongrightarrow{\aDel} #2}

% Form
\newcommand{\farr}{\mathtt{arrow}}
\newcommand{\fnum}{\mathtt{num}}

\newcommand{\fasc}{\mathtt{asc}}
\newcommand{\fvar}[1]{\mathtt{var}~#1}
\newcommand{\flam}[1]{\mathtt{lam}~#1}
\newcommand{\fap}{\mathtt{ap}}
\newcommand{\farg}{\mathtt{arg}}
\newcommand{\fnumlit}[1]{\mathtt{numlit}~#1}
\newcommand{\fplus}{\mathtt{plus}}
\newcommand{\fhole}{\mathtt{hole}}

% Talk about formal rules in example
\newcommand{\refrule}[1]{\textrm{Rule~(#1)}}

\begin{document}

\conferenceinfo{-}{-}
\copyrightyear{-}
\copyrightdata{[to be supplied]}

%\titlebanner{}        % These are ignored unless
\preprintfooter{Draft}   % 'preprint' option specified.

\title{Hazelnut: A Bidirectionally Typed \\ Structure Editor Calculus}
%\subtitle{Modular Type Constructors}

%\authorinfo{\vspace{-2px}}{}{}
\authorinfo{~}{~}{~}
%\authorinfo{Cyrus Omar \and Jonathan Aldrich}
%         {Carnegie Mellon University}
%         {\{comar, aldrich\}@cs.cmu.edu}

\maketitle
\begin{abstract}
Well-typed expressions are rich inductive structures, but programmers typically construct and manipulate them only indirectly, as text that must pass through a parser and typechecker. Not all text survives this journey. In particular, text that arises transiently during the editing process, or when the programmer has made a mistake, is often malformed or ill-typed. Contending with (i.e. reasoning about) malformed text or well-formed but meaningless expressions is a difficult task, for programmers and their tools alike.

%  \emph{Structure editors} have long promised to alleviate these burdens by exposing only edit actions that  cause sensible changes to the program structure.
% Existing designs for structure editors, however, are complex and somewhat \emph{ad hoc}. They also focus primarily on syntactic well-formedness, so programs can still be left semantically meaningless as they are being constructed.

We introduce Hazelnut, a minimal \emph{structure editor} where every edit action leaves the expression being edited in both a statically and dynamically meaningful state (with the latter implying that Hazelnut is particularly suitable as a basis for \emph{live programming}.) Uniquely, 1) Hazelnut is defined in a principled type-theoretic style, with a rich metatheory mechanized in Agda; and 2) users need not construct outer forms before they construct inner forms. Instead, Hazelnut's \emph{action semantics} automatically places terms that have a type that is inconsistent with the expected type into a \emph{hole}, which safely defers the type consistency check until the hole is \emph{finished}. Interestingly, this mechanism is related to a mechanism that arises in the semantics of gradual type systems. %Formally, Hazelnut is a bidirectionally typed lambda calculus extended with \emph{holes}, a \emph{focus model} (based on Huet's zipper) and an \emph{action model}.
\end{abstract}

%\category{D.3.2}{Programming Languages}{Language Classifications}[Extensible Languages]
%\category{D.3.4}{Programming Languages}{Processors}[Compilers]
%\category{F.3.1}{Logics \& Meanings of Programs}{Specifying and Verifying and Reasoning about Programs}[Specification Techniques]
%\keywords
%extensible languages; module systems; type abstraction; typed compilation; type-level computation

\section{Introduction}\label{sec:introduction}
% !TEX root = hazelnut-popl17.tex

%% There are some benefits to this approach, to be sure, but the structural
%% mismatch between programs and their textual representations also imposes
%% various burdens.  For example, the primitive edit actions available in a
%% text editor (e.g. inserting or deleting a character or word) do not
%% always correspond to sensible structural transformations.

% spj: describe the problem; state our contributions; STOP. one page max.

% When constructing a program or proof in a language with rich type
% structure, skilled programmers generally follow a \emph{type discipline}
% where they first determine the type of the expression that they are
% constructing in order to constrain the mental search space that they are
% operating within.

% For example, if the programmer knows that an expression of type
% $\tarr{\tnum}{\tnum}$ is needed, then it is often the case (though, of course, not
% necessarily the case) that the expression will take the form $$\hlam{\mathit{x}}{e}$$
% for some variable $x$ and function body $e$. If the programmer chooses this
% form, then after picking a suitable variable name, her focus will be on
% constructing a suitable body, $e$. Following the type discipline, $e$ must
% be of type $\tnum$, and so this process can begin anew.

% The problem is that when using a text
% editor to construct a program, it is easy, and indeed necessary, to deviate from this disciplined process.  Rather, text editors operate on sequences of
% characters (i.e. \emph{text}.) 

%Although programs can be represented as text, most text does not correspond to a syntactically well-formed and semantically well-defined program. 
% Programming languages, and therefore programs themselves, are rich mathematical structures.
Programmers typically construct and manipulate well-typed expressions only indirectly, by editing text that is first parsed according to a textual syntax and then typechecked according to a static semantics. This indirection has practical benefits, to be sure -- text editors and other text-based tools benefit from decades of development effort -- but it also introduces some fundamental complexity into the programming process. 

% Every day, programmers use text-based tools to construct and manipulate programs.  
% These programs are written in languages which defined by a textual syntax. 
% While textual syntax has proved to have practical utility, it also introduces some fundamental complexity into the programming process.

% It also complicates matters for tool designers because tools are often confronted with text that does not correspond to a well-formed, meaningful program\todo{mention YoungSeok's data here}. This may be because the programmer is in the midst of a sequence of 
% edit actions that leaves the text temporarily malformed or ill-typed, or because the programmer has made a mistake. The language definition is silent about these situations, so it is difficult for tools to help programmers determine and execute a corrective course of action (e.g. by providing \todo{cite study on benefits of syntax highlighting}syntax highlighting and semantics-aware code completion services\todo{cite something?}.)
%Doesn't really start out with a bang, IMHO. I like to start papers out with a "big" problem.
%Most programming languages define a textual syntax. This allows programmers to use text editors and other standard text-based tools to construct and manipulate programs. While this is of substantial practical utility, it also introduces some fundamental complexity into the programming process. 

First, it requires that programmers understand the various subtleties of the textual syntax (e.g. the relative precedence of the various forms.) This can be particularly challenging for novice programmers \cite{Altadmri:2015:MCI:2676723.2677258}. %For example, one study of novice programmers found syntax errors (e.g. unbalanced parentheses) to be the most common class of errors.
% Too much detail? 

The fact that not every sequence of characters corresponds to a meaningful program also complicates matters for tool designers. In particular, program editors must contend with meaningless text on a regular basis: in a dataset collected by Yoon and Myers consisting of 1460 hours of fine-grained edit logs from 21 programmers, 44.2\% of edits resulted in a malformed edit state \cite{6883030}. Some additional percentage of edit states were well-formed but ill-typed (the dataset did not gather the imported libraries, so an exact percentage could not be determined.)  
Some of these edit states arose because the programmer was in the midst of a sequence of edits that left the edit state temporarily malformed or ill-typed, while others could be attributed to programmer error. The language definition is silent about these edit states, so it is difficult to design useful editor services (e.g. syntax highlighting~\cite{sarkar2015impact}, type-aware code completion~\cite{Mooty:2010:CCC:1915084.1916348,Omar:2012:ACC:2337223.2337324}, and refactoring support \todo{citation}), and to report localized, meaningful error messages\todo{citations}.\todo{heuristics}
%, or because the programmer has made a mistake. 
% The syntax definition is silent about these situations, so it is difficult for tools to help programmers determine and execute a corrective course of action 

% Many editors have developed \emph{ad hoc} workarounds for this problem, e.g. they might attempt to use regular expressions to highlight malformed program text, insert closing delimiters automatically, use whitespace to guess where a delimiter is likely to appear, or continue past a type error by pretending that the type was as expected (if, indeed, an expected type can be determined.) These heuristic methods are often complex, and they can confuse or mislead the programmer. % This may also help explain why the error messages emitted by parsers and typecheckers are often quite baroque\todo{citations}.

% stuck editing a representation of the program instead of the structures
% themselves. The editor does not restrict what the programmer may do: you
% can delete characters that belong, insert ones that don't, forget things
% that were needed, and so. There's nothing stopping us from accidentally
% writing $$\lambda \mathit{x:num}.\mathit{(x,x)}$$ even though it's obvious
% that building a pair can't hope to form a natural number.

% The type structure of the language makes this sort of error obvious: it's
% not that you're adding characters that make your program incorrect, or even
% malformed; you're adding characters that can't possibly create a structure
% you want because of the type. Simply put, the primitive operations
% available in text editors do not always correspond to sensible
% transformations on the structure of the program.
These complications have motivated a long line of research into \emph{structure editors}, i.e. program editors where the edit state is a  program structure, rather than text. % Eliminating text eliminates the possibility of syntax errors.

Most structure editors are \emph{syntactic structure editors}, i.e. the edit state is a syntax tree with \emph{holes} that stand for branches of the tree that have yet to be constructed, and the edit actions are context-free tree transformations. For example, Scratch is a syntactic structure editor that has achieved success as a tool for teaching children how to program \cite{Resnick:2009:SP:1592761.1592779}. 

The Scratch language has a trivial static semantics, but researchers have also designed syntactic structure editors for  languages with a non-trivial static semantics. For example, \texttt{mbeddr} is an editor for a C-like language \cite{voelter_mbeddr:_2012}, TouchDevelop is an editor for an object-oriented language \cite{tillmann_touchdevelop:_2011} and Lamdu is an editor for a functional language similar to Haskell \cite{lamdu}. Each of these editors presents an innovative user interface, but the non-trivial type and binding structure of the underlying language poses several challenges that complicates its design. 
% \begin{itemize}
% \item 

First, there is an all too familiar challenge: these languages do not have a clear formal semantics and metatheory, i.e. ``the implementation is the specification''. This makes it difficult to reason methodically about types and binding, so, in turn, it is difficult to design reliable editor services.

Even if a formal semantics and metatheory for the underlying language were forthcoming (e.g. following Standard ML \cite{Harper00atype-theoretic,mthm97-for-dart}), it would go only part of the way towards addressing the challenges faced by structure editor designers. First, a standard semantics assigns no formal meaning to \emph{incomplete terms}, i.e. terms with holes. Moreover, a syntactic structure editor does not guarantee that every edit state is statically meaningful -- only that it is syntactically well-formed -- so incorporating holes into the static semantics would in any case be only a minor salve. Heuristics analagous to those that pervade textual program editors would still be needed to contend with the ill-typed edit states.

This paper seeks to address these foundational challenges. In particular, we introduce Hazelnut,  a \emph{typed structure editor}  organized around a bidirectionally typed lambda calculus extended to assign static meaning to expressions and types with {holes}. Hazelnut's formal \emph{action semantics} maintains the invariant that every edit state is a statically meaningful (i.e. well-typed) expression with a single superimposed \emph{focus} (following Huet's \emph{zipper pattern} \cite{JFP::Huet1997}.) Actions act relative to the expression or type in focus (which need not be a hole.)  %More specifically, actions are context-aware,  acting on an expression whose type is determined by its surroundings (e.g. a function argument), only actions consistent with that type are permitted. 

Na\"ively, enforcing this injunction on ill-typed edit states would force programmers to construct programs in a rigid ``outside-in'' manner. For example, the programmer would often need to construct the outer function application form before identifying the intended function. To avoid this problem, Hazelnut automatically places newly constructed expressions inside a hole if the expression's type is inconsistent with the expected type. This safely defers the type consistency check until the expression inside the hole is \emph{finished}. In other words, holes appear both at the leaves and at the internal nodes of the syntax tree that remain under construction. %In short, Hazelnut is a \emph{bidirectionally typed structure editor calculus}. %Actions act at a programmer-indicate subtree, called the \emph{focus} (which is defined following Huet's zipper pattern.) 

% By defining \emph{holes} as a language construct, for both expressions as types, Hazelnut enables type-aware actions that always leave the program in both a structurally and semantically well-defined state. 
%In Hazelnut, expressions and types with \emph{holes} have a well-defined static semantics. Edit actions are type-aware and leave the program in both a structurally and semantically well-defined state. 



%However, these syntactically correct states can be semantically meaningless (i.e. undefined), because the language definitions generally only give meaning to complete, well-typed terms. (e.g., when a branching element is introduced, the program does not become semantically correct until both branches are complete)
% This makes it difficult for humans and tools to reason about types and binding during the development process, even when using a structure editor.
% Similar to text editors, structure editors also can develop workarounds to try to help users, but these efforts can only extend so far, because of the underlying language definition.
%Some structure editors attempt stuff, but it is not clear what invariants are being maintained (e.g. Unison ; Scratch seems not to allow literals of the wrong type, but variables are not typed.)\todo{revise} %More sophisticated semantic reasoning principles thus remain .


 



The remainder of the paper is organized as follows:
\begin{itemize}[noitemsep,nolistsep]
\item We begin in Section
    \ref{sec:example} with two example edit sequences to develop the reader's intuitions.  
\item We then give a detailed overview of Hazelnut's semantics and metatheory, which has been mechanized in Agda, in Section \ref{sec:hazel}. 
\item Hazelnut is designed as a {foundational} calculus, i.e. as a calculus that language and editor designers extend with higher level constructs. We show how Hazelnut's rich mechanized metatheory guides and constrains the development of such an extension in Section \ref{sec:extending}. 
\item In Section \ref{sec:impl}, we briefly describe how Hazelnut's action semantics lends itself to efficient implementation as a functional reactive program. Our reference implementation is written using \lstinline{js_of_ocaml} and the OCaml \lstinline{React} library.
 
\item We consider several possible evaluation strategies for incomplete expressions in Section \ref{sec:dynamics}. In so doing, we discover interesting connections with gradual typing and contextual modal type theory. The former provides an interpretation of type holes, and the latter provides a logical interpretation of expression holes. It also suggests a principled design for an ``edit and resume'' feature. 

\item In Section \ref{sec:rw}, we summarize the related work.
\item We conclude in Section \ref{sec:future} by summarizing our vision of a principled science of structure editing rooted in type theory, and suggest a number of future directions.
\end{itemize} 

\begin{figure}[t!]
\[
\begin{array}{|c||c|c||l|l|}
\hline
\# & \textbf{Z-Expression} & 
%\textbf{H-Expression} & 
% \textbf{Type} & 
\textbf{Next Action} & \textbf{Rule}
\\
\hline
1 &
\zwsel{\hhole{}} & 
% \hhole{} &
% \tehole 
% &
\aConstruct{\flam{x}} & 
\text{(\ref{r:conelamhole})}
\\ 2 &
\hlam{x}{\hhole{}} : \tarr{\zwsel{\hhole{}}}{\hhole{}} & 
% \hlam{x}{\hhole{}} : \tarr{\hhole{}}{\hhole{}} &
% \tarr{\tehole}{\tehole} &
\aConstruct{\fnum{}} &
\text{(\ref{r:contnum})}
\\ 3 &
\hlam{x}{\hhole{}} : \tarr{\zwsel{\tnum{}}}{\hhole{}} &
% \hlam{x}{\hhole{}} : \tarr{\tnum{}}{\hhole{}} &
% \tarr{\tnum}{\tehole} & 
\aMove{\dNext{}} & 
\text{({\ref{r:movenextsib}})}
\\ 4 &
\hlam{x}{\hhole{}} : \tarr{\tnum}{\zwsel{\hhole{}}}
&
% \text{\textquotedbl}&
% \tarr{\tnum}{\tehole} & 
\aConstruct{\fnum{}} & 
\text{(\ref{r:contnum})}
\\ 5 &
\hlam{x}{\hhole{}} : \tarr{\tnum{}}{\zwsel{\tnum{}}} & 
% \hlam{x}{\hhole{}} : \tarr{\tnum{}}{\tnum{}} &
% \tarr{\tnum}{\tnum} &
\aMove{\dParent{}} & 
\text{(\ref{r:moveparent})}
\\ 6 &
\hlam{x}{\hhole{}} : \zwsel{\tarr{\tnum{}}{\tnum{}}}
&
% \text{\textquotedbl}&
% \tarr{\tnum}{\tnum} & 
\aMove{\dPrev{}} & 
\text{(\ref{r:moveprevsib})}
\\ 7 &
% &
\zwsel{\hlam{x}{\hhole{}}} : \tarr{\tnum{}}{\tnum{}} & 
% \tarr{\tnum}{\tnum} &
% &
\aMove{\dChild{}} & 
\text{(\ref{r:movefirstchild-lam})}
\\ 8 &
% &
\hlam{x}{\zwsel{\hhole{}}} : \tarr{\tnum{}}{\tnum{}} &
% \tarr{\tnum}{\tnum} & 
\aConstruct{\fvar{x}} & 
\text{(\ref{r:conevar})}
\\ 9 &
% \hlam{x}{{x}} : \tarr{\tnum{}}{\tnum{}}
% &
\hlam{x}{\zwsel{{x}}} : \tarr{\tnum{}}{\tnum{}} &
% \tarr{\tnum}{\tnum} & 
\quad\textrm{---}
&
\quad\textrm{---}
\\
\hline
% \multicolumn{5}{c}{\text{... now assume a context where $id : \tarr{\tnum}{\tnum}$ ...}}\\
% \hline
% 10 &
% % \hhole{} &
% \zwsel{\hhole{}} & 
% \tehole &
% \aConstruct{\fasc} & 
% \text{(\ref{r:constructasc})}
% \\
% 11 &
% % \hhole{} : \hhole{} &
% \hhole{} : \zwsel{ \hhole{}} &
% \tehole & 
% \aConstruct{\fnum{}} & 
% \text{(\ref{r:contnum})}
% \\
% 12 &
% % \hhole{} :\tnum{} &
% \hhole{} : \zwsel{\tnum{}} &
% \tnum &
% \aMove{\dPrev{}} & 
% \text{(\ref{r:moveprevsib})}
% \\
% % %13 &
% % %\hhole{} :\tnum{} &
% % %\zwsel{\hhole{}} : \tnum{}
% % %&
% % %\aMove{\dPrev{}} & \refrule{\ref{r:moveprevsib}}
% % %\\
% 13 &
% % \hhole{} :\tnum{}
%  % &
% \zwsel{\hhole{}} : \tnum{} &
% \tnum & 
% \aConstruct{ \fvar{id}} & 
% \text{(\ref{r:conevar2})}
% \\
% 14 &
% % \hhole{\textrm{$id$}} : \tnum{} &
% \hhole{\zwsel{{\textrm{$id$}}}} : \tnum{} &
% \tnum & 
% \aConstruct{\fap{}} & 
% \text{(\ref{r:coneapfn})}
% \\
% 15 &
% % \hhole{\hap{{{\textrm{$id$}}}}{{\hhole{}}}} : \tnum{}
% % &
% \hhole{\hap{{{\textrm{$id$}}}}{\zwsel{\hhole{}}}} : \tnum{} &
% \tnum & 
% \aConstruct{\fnumlit{3}} &  
% \text{(\ref{r:conenumnum})}
% \\
% 16 &
% % \hhole{\hap{{{\textrm{$id$}}}}{{\hnum{3}}}} : \tnum{}
% % &
% \hhole{\hap{{{\textrm{$id$}}}}{\zwsel{\hnum{3}}}} : \tnum{} &
% \tnum & 
% \aMove{\dParent{}} &  
% \text{(\ref{r:moveparent-ap2})}
% \\
% 17 &
% % %\hhole{\hap{{{\textrm{id}}}}{{\hnum{3}}}} : \tnum{}
% % &
% \hhole{\zwsel{\hap{{{\textrm{$id$}}}}{{\hnum{3}}}}} : \tnum{} &
% \tnum & 
% \aMove{\dParent{}} &  
% \text{(\ref{r:moveparent-hole})}
% \\
% 18 &
% % %\hhole{\hap{{{\textrm{id}}}}{{\hnum{3}}}} : \tnum{}
% % &
% \zwsel{\hhole{{\hap{{{\textrm{$id$}}}}{{\hnum{3}}}}}} : \tnum{} &
% \tnum & 
% \aFinish &  
% \text{(\ref{r:finishana})}
% \\
% 19 &
% % {\hap{{{\textrm{$id$}}}}{{\hnum{3}}}} : \tnum{}
% % &
% \zwsel{{{\hap{{{\textrm{$id$}}}}{{\hnum{3}}}}}} : \tnum{} &
% \tnum & 
% \quad\textrm{---} & 
% \quad\textrm{---}
% \\
% \hline

%% 11 &
%% {\textrm{id}} : \tarr{\tnum{}}{\tnum{}}
%% &
%% \hlam{x}{\zwsel{{x}}} : \tarr{\tnum{}}{\tnum{}}
%% &
%% \aMove{\dParent{}} & ?
%% \\ 12 &
%% %{\textrm{id}} : \tarr{\tnum{}}{\tnum{}}
%% &
%% \zwsel{\hlam{x}{{{x}}}} : \tarr{\tnum{}}{\tnum{}}
%% &
%% \aMove{\dParent{}} & \refrule{15b} bad
%% \\ 13 &
%% %{\textrm{id}} : \tarr{\tnum{}}{\tnum{}}
%% &
%% \zwsel{\hlam{x}{{{x}}} : \tarr{\tnum{}}{\tnum{}}}
%% &
%% \aConstruct{\fap{}} & \refrule{20h} bad
%% \\ 14 &
%% \hapP{{\textrm{id}} : \tarr{\tnum{}}{\tnum{}}}{\hhole{}}
%% &
%% \hapP{{\textrm{id}} : \tarr{\tnum{}}{\tnum{}}}{\zwsel{\hhole{}}}
%% &
%% \aConstruct{\fnumlit{3}} & \refrule{20l} bad
%% \\ 15 &
%% %\hapP{{\textrm{id}} : \tarr{\tnum{}}{\tnum{}}}{\hhole{3}}
%% %&
%% %\hapP{{\textrm{id}} : \tarr{\tnum{}}{\tnum{}}}{\zwsel{\hhole{3}}}
%% %&
%% %\aFinish{} & ?
%% %\\
%% \hapP{{\textrm{id}} : \tarr{\tnum{}}{\tnum{}}}{{\hnum{3}}}
%% &
%% \hapP{{\textrm{id}} : \tarr{\tnum{}}{\tnum{}}}{\zwsel{{\hnum{3}}}}
%% &
%% \quad\textrm{---{}---}
%% &
%% \quad\textrm{---{}---}
%% \\
%% \hline
\end{array}
\]
\caption{Constructing an identity function in Hazelnut. The formal syntax and the rules referenced in the final column are described in Section \ref{sec:hazel}.}
\label{fig:first-example}
\end{figure}

\section{Programming in Hazelnut}\label{sec:example}
% !TEX root = hazelnut-popl17.tex

%
\todo{add let and ifz}\todo{add let to example}\todo{finalize table}Figure~\ref{fig:first-example} gives an example of the Hazelnut user
performing two simple programming tasks.
The syntactic forms in this figure will be formally defined in Sec. \ref{sec:hazel}. For now, we will develop only the necessary intuitions. In the first task (Lines 1-9), the user constructs the identity function over numbers. In the second task (Lines 10-19), the user applies this function (assumed to be bound to a variable, $id$), to the number expression $\hnum{3}$.
Each of these tasks is carried out interactively, through the sequence of \emph{actions} shown in the  column labeled \textbf{Next Action}. For reference, we cite the relevant rules from Sec. \ref{sec:hazel} in the final column.

The second and third columns of the
table show the program as it is being constructed in two forms. The second column shows it as an \textbf{H-expression}, which is an expression that can contain \emph{holes}, delimited by $\llparenthesis$ and $\rrparenthesis$. The third column shows a corresponding \textbf{Z-expression}. Z-expressions are H-expressions with a single focus on some sub-term, delimited by $\triangleright$ and $\triangleleft$. The focus need not be on a hole.
% on working on filling just one of the holes.
Each action produces a new Z-expression, but this may or may not correspond to a new H-expression (in particular, some actions only move the focus, without changing the structure of the term.)
% to the hole in
%focus in the Z-Expression to produce the next line, which may or may not
%produce a substantively different H-Expression.

Line~1 begins with the simplest initial expression: an H-expression
consisting of a single hole. The corresponding Z-Expression has that hole in focus,
indicated by the syntax~$\zwsel{\hehole}$. Focus determines the locus of action. The first action the user performs is $\aConstruct{\flam{x}}$, which replaces the hole with a lambda abstraction binding the variable $x$. This results in the program on line
2, consisting of a lambda abstraction ascribed an arrow type with holes in all positions. The argument type hole is in focus. The
user proceeds to fill these holes using construction and movement actions, resulting in the final expression on Line 9. With no holes remaining, this expression is \emph{complete}.% (though it could, of course, undergo further actions nevertheless.)

So far, editing has proceeded in an essentially type-directed, outside-in fashion -- the user first specified the type of the function, then produced a body of that type by the action on Line 8. Lines 10-12 similarly begin in a type-directed manner with the user giving an explicit type ascription, indicating that the expression that they are constructing will have type $\tnum$. 

However, on Line 13, the user performs the $\aConstruct{\fvar{id}}$ action. Notice that $id$ has type $\tarr{\tnum}{\tnum}$, which is not consistent with the type $\tnum$ given in the ascription. Na\"ively, this would produce a type error, leaving the program in a well-formed but semantically undefined state. One way to avoid this state is to simply not make this action available in the program configuration on Line 12. This is inflexible, forcing an outside-in approach to program construction (i.e. the user would need to construct the function application form before constructing the variable $id$.) Instead, Hazelnut permits this action, but places the variable $id$ inside a hole. This defers the consistency check that would normally occur: a hole can be checked against any type, as long as its contents have some type. The cursor is placed inside the hole. The user then proceeds to apply $id$ to the number expression $\hnum{3}$. At this point, the expression inside the hole has a type consistent with the ascription, so the user can \emph{finish} the hole. In our simple formalism, this requires moving the cursor to the hole (in practice, the system might find the nearest parent of hole form.) The result is the complete, well-typed program shown on Line 19 (notice that \emph{complete} is distinct from \emph{closed} -- the variable $id$ is free on Line 19, so this is not a closed program.)

%% The third column~(\textbf{Next Action}) lists the first user action:
%% Constructing a lambda abstraction using variable~$x$.
%% %
%% The final column~(\textbf{Semantics}) indicates the semantic rule for this
%% action, Rule (\ref{r:conelamhole}), which gives general semantics for
%% introducing lambda terms into holes.
%% %
%% In Section~\ref{sec:hazel}, we list this rule, and the other rules used in
%% this final column. In total, these rules give a formal semantics to the
%% user actions, which relate each line's Z-Expression to the Z-Expression on
%% the subsequent line.

%% In addition to introducing the lambda term, and its variable, the
%% first user action~$\aConstruct{\flam{x}}$ also introduces a type
%% ascription for this function, as an arrow type, with holes for the
%% type of its domain and codomain.
%% %
%% The actions for Lines~2--5 consist of the user filling these holes
%% with the basetype $\tnum{}$.
%% %
%% To do so, the user constructs the type constructor twice (Lines 2 and
%% 4), and navigates between the holes with a move action (Line~3).
%% %
%% Generally, the move action~$\dNext$ moves the focus from one
%% sub-structure to the next sibling sub-structure of the (common) parent
%% structure; in this case, it moves from the domain type of the arrow
%% type to the codomain of the arrow type.
%% %

\section{Hazelnut, Formally}
\label{sec:hazel}
Hazelnut is based on the simply-typed lambda calculus extended with a single base type, $\tnum$. Its major constituents, introduced by example in the previous section, are:
\begin{itemize}
\item \textbf{H-types} and \textbf{H-expressions} (Sec. \ref{sec:holes}), which are terms with \emph{holes}. Holes mark subterms that are ``under construction.'' H-types classify H-expressions according to a {bidirectionally typed} static semantics.
\item \textbf{Z-types} and \textbf{Z-expressions} (Sec. \ref{sec:cursors}), which superimpose a single \emph{focus} onto H-types and H-expressions (using Huet's \emph{zipper pattern} \cite{JFP::Huet1997}.)
\item \textbf{Actions} (Sec. \ref{sec:actions}), which move the focus or modify the subterm in focus.

Whenever an action is performed on a well-typed expression, it produces another well-typed expression in a \emph{sensible} manner. More specifically, the action semantics satisfies a crucial \emph{sensibility theorem}, stated in Sec. \ref{sec:actions}.
\end{itemize}

In our overview of the semantics below, we will reproduce only the most interesting rules, and in some cases we will do so ``out of order.'' The appendix (and our Agda formalization, see Sec. \ref{sec:mech}) defines the complete collection of rules in their dependency order.
\subsection{Holes}\label{sec:holes}
\begin{figure}[t]
$\arraycolsep=4pt\begin{array}{lllllll}
\mathsf{HTyp} & \tau,\htau & ::= &
  \tarr{\htau}{\htau} ~\vert~
  \tnum ~\vert~
  \tehole\\
\mathsf{HExp} & e,\hexp & ::= &
  \hexp : \htau ~\vert~
  x ~\vert~
  \hlam{x}{\hexp} ~\vert~
  \hap{\hexp}{\hexp} ~\vert~
  \hnum{n} ~\vert~
  \hadd{\hexp}{\hexp} ~\vert~
  \hehole ~\vert~
  \hhole{\hexp}
\end{array}$
%\textbf{Sort} & & & \textbf{Operational Form} & \textbf{Stylized Form} & \textbf{Description}\\
\caption{Syntax of H-types and H-expressions. Metavariable $x$ ranges over variables and $n$ ranges over numerals.}
\label{fig:hexp-syntax}
\end{figure}

The syntax of H-types and H-expressions is given in Figure \ref{fig:hexp-syntax}. Most of the forms correspond directly to those of the simply-typed lambda calculus extended with type $\tnum$. The number expression corresponding to the number $n$ is drawn $\hnum{n}$, and for simplicity, we define only a single arithmetic operation, $\hadd{\hexp}{\hexp}$.
In addition to these standard forms, \emph{empty holes} are drawn $\hehole$ and \emph{non-empty H-expression holes} are drawn $\hhole{\hexp}$. In our simple calculus, all well-formed type expressions are valid types, so we do not need non-empty H-type holes.%Holes mark subterms that are, notionally, ``under construction.'' We will see what this formally corresponds to in a moment.

We refer to terms that do not contain subterms of hole form as \emph{complete}. Informally, we will use metavariables $\tau$ and $e$ rather than $\htau$ and $\hexp$ for complete H-types and H-expressions, respectively. Formally, we can derive $\hcomplete{\tau}$ when $\tau$ is a complete H-type, and $\hcomplete{e}$ when $e$ is a complete H-expression. We omit the straightforward definitions of these judgements for concision. The dynamics of Hazelnut, which we need not detail here, is defined only  over complete H-expressions (i.e. we can only ``run'' a complete program, though see Sec. \ref{sec:future}.)

The statics of Hazelnut is organized as a \emph{bidirectional type system} \cite{Pierce:2000:LTI:345099.345100}, i.e. around the following mutually defined typing judgements:
\[\arraycolsep=15pt\begin{array}{ll}
%\textbf{Judgement Form} & \textbf{Description}\\
\hana{\hGamma}{\hexp}{\htau} & \text{$\hexp$ analyzes against $\htau$}\\
\hsyn{\hGamma}{\hexp}{\htau} & \text{$\hexp$ synthesizes $\htau$}
\end{array}\]
where typing contexts, $\hGamma$, map each variable $x \in \domof{\hGamma}$ to a hypothesis $x : \htau$.
Derivations of the type analysis judgement establish that $\hexp$ can appear where an expression of type $\htau$ is expected. Derivations of the type synthesis judgement determine a type that can be assigned to $\hexp$ even in positions where an expected type is not known (e.g. at the top level.) Algorithmically, the type is an ``input'' of the type analysis judgement, but an ``output'' of the type synthesis judgement. %The rules describe a \emph{local type inference} scheme, i.e. type ascriptions are unnecessary when an expression is being analyzed against a known type.
Making a judgemental distinction between these two notions will be essential for giving a sensible action semantics to our system (Sec. \ref{sec:actions}.)

 %We use the metavariable $\Gamma$ for \emph{complete typing contexts}, i.e. typing contexts where each hypothesis mentions only complete types.


\begin{subequations}\label{rules:syn-ana}
Type synthesis is stronger than type analysis in that if an expression is able to synthesize a type, it can also be analyzed against that type, or any \emph{compatible} type. This is expressed by the \emph{subsumption rule}:
\begin{equation}\label{rule:ana-subsume}
\inferrule{
  \hsyn{\hGamma}{\hexp}{\htau'}\\
  \tcompat{\htau}{\htau'}
}{
  \hana{\hGamma}{\hexp}{\htau}
}
\end{equation}
The \emph{H-type compatibility judgement}, $\tcompat{\htau}{\htau'}$, reduces to syntactic equality for complete H-types. For incomplete H-types, the rules are given after we discuss the semantics of holes below.

First, let us briefly review the standard constructs.
Type ascription allows the user to state  a type for the ascribed expression to be analyzed against:
\begin{equation}\label{rule:syn-asc}
\inferrule{
  \hana{\hGamma}{\hexp}{\htau}
}{
  \hsyn{\hGamma}{\hexp : \htau}{\htau}
}
\end{equation}

A variable synthesizes the type that the context assigns to it:
\begin{equation}\label{rule:syn-var}
\inferrule{ }{
  \hsyn{\hGamma, x : \htau}{x}{\htau}
}
\end{equation}

Functions are not themselves annotated with types, so they can only appear in analytic position:
\begin{equation}\label{rule:syn-lam}
\inferrule{
  \hana{\hGamma, x : \htau_1}{\hexp}{\htau_2}
}{
  \hana{\hGamma}{\hlam{x}{\hexp}}{\tarr{\htau_1}{\htau_2}}
}
\end{equation}
(It would be straightforward to also add a ``half-annotated'' lambda form, $\lambda x{:}\tau.e$, but for simplicity, we leave it out of our calculus \cite{DBLP:conf/tldi/ChlipalaPH05}.)

For function application, if the expression in function position synthesizes an arrow type, the argument is analyzed against the synthesized argument type:
\begin{equation}\label{rule:syn-ap}
\inferrule{
  \hsyn{\hGamma}{\hexp_1}{\tarr{\htau_2}{\htau}}\\
  \hana{\hGamma}{\hexp_2}{\htau_2}
}{
  \hsyn{\hGamma}{\hap{\hexp_1}{\hexp_2}}{\htau}
}
\end{equation}

Numbers synthesize type $\tnum$:
\begin{equation}\label{rule:syn-num}
\inferrule{ }{
  \hsyn{\hGamma}{\hnum{n}}{\tnum}
}
\end{equation}

Addition operates like a function over numbers:
\begin{equation}\label{rule:syn-plus}
\inferrule{
  \hana{\hGamma}{\hexp_1}{\tnum}\\
  \hana{\hGamma}{\hexp_2}{\tnum}
}{
  \hsyn{\hGamma}{\hadd{\hexp_1}{\hexp_2}}{\tnum}
}
\end{equation}

The rules given so far are sufficient to type complete H-expressions. The remaining rules give H-expressions with holes a well-defined static semantics.

The empty hole synthesizes the hole type:
\begin{equation}\label{rule:syn-ehole}
\inferrule{ }{
  \hsyn{\hGamma}{\hehole}{\tehole}
}
\end{equation}

A non-empty hole contains an H-expression that is ``under construction''. The inner expression must synthesize some type, but the non-empty hole synthesizes only the hole type:
\begin{equation}\label{rule:syn-hole}
\inferrule{
  \hsyn{\hGamma}{\hexp}{\htau}
}{
  \hsyn{\hGamma}{\hhole{\hexp}}{\tehole}
}
\end{equation}
The type compatibility judgement $\tcompat{\htau}{\htau'}$, which appeared as a premise in the subsumption rule, makes the hole type compatible with any other type:
\begin{subequations}\label{rules:tcompat}
\begin{equation}\label{rule:tcompat-hole}
\inferrule{ }{
  \tcompat{\htau}{\tehole}
}
\end{equation}
The remaining rules, given in the appendix, establish that type compatibility is symmetric and reflexive (but not transitive.)
% \begin{equation}\label{rule:tcompat-comm}
% \inferrule{
%   \tcompat{\htau}{\htau'}
% }{
%   \tcompat{\htau'}{\htau}
% }
% \end{equation}
% \begin{equation}\label{rule:tcompat-num}
% \inferrule{ }{
%   \tcompat{\tnum}{\tnum}
% }
% \end{equation}
% \begin{equation}\label{rule:tcompat-arr}
% \inferrule{
%   \tcompat{\htau_1}{\htau_1'}\\
%   \tcompat{\htau_2}{\htau_2'}
% }{
%   \tcompat{\tarr{\htau_1}{\htau_2}}{\tarr{\htau_1'}{\htau_2'}}
% }
% \end{equation}
\end{subequations}
Consequently, by subsumption, we can derive that $\hana{id : \tarr{\tnum}{\tnum}}{\hhole{id}}{\tnum}$, as is necessary to synthesize a type for the H-expression on Line 14 of Fig. \ref{fig:first-example}. %In other words, this mechanism is essential if  users are to able to construct a program in anything but an ``outside in'' fashion.

The final rule handles function applications where the expression in function position synthesizes a hole type, rather than an arrow type. We treat it as if it had instead synthesized $\tarr{\tehole}{\tehole}$:
\begin{equation}\label{rule:syn-ap-2}
\inferrule{
  \hsyn{\hGamma}{\hexp_1}{\tehole}\\
  \hana{\hGamma}{\hexp_2}{\tehole}
}{
  \hsyn{\hGamma}{\hap{\hexp_1}{\hexp_2}}{\tehole}
}
\end{equation}

The hole type behaves much like the type $?$ in prior work by Siek and Taha on gradual types for functional languages \cite{Siek06a}. Their system (which was not bidirectionally typed nor an editor model) also needed to define two rules for function application. In general, when a premise requires that a synthesized type be of a particular form, we need a special case where the synthesized hole type is treated instead as if it were the ``holey-est'' type of that form.\footnote{Alternatively, we might add a rule that allows expressions that synthesize hole type to then non-deterministically synthesize any other type, but maintaining determinism is useful in practice, so we avoid this approach.}

\end{subequations}
\subsection{Focus Model}\label{sec:cursors}
\begin{figure}[t]
\hspace{-3px}$\arraycolsep=3pt\begin{array}{lllllll}
\mathsf{ZTyp} & \ztau & ::= &
  %\zlsel{\htau} ~\vert~
  \zwsel{\htau} ~\vert~
  %\zrsel{\htau} ~\vert~
  \tarr{\ztau}{\htau} ~\vert~
  \tarr{\htau}{\ztau} \\
\mathsf{ZExp} & \zexp & ::= &
  %\zlsel{\hexp} ~\vert~
  \zwsel{\hexp} ~\vert~
  %\zrsel{\hexp} ~\vert~
  \zexp : \htau ~\vert~
  \hexp : \ztau ~\vert~
  \hlam{x}{\zexp} ~\vert~
  \hap{\zexp}{\hexp} ~\vert~
  \hap{\hexp}{\zexp} ~\vert~
  \hadd{\zexp}{\hexp} ~\vert~
  \hadd{\hexp}{\zexp} ~\vert~
  \hhole{\zexp}
\end{array}$
%\textbf{Sort} & & & \textbf{Operational Form} & \textbf{Stylized Form} & \textbf{Description}\\
\caption{Syntax of Z-types and Z-expressions, i.e. types and expressions with holes and a single focus.}
\label{fig:zexp-syntax}
\end{figure}

In order to identify a single subtree of an H-type or H-expression as the current focus of action, we apply Huet's \emph{zipper pattern} \cite{JFP::Huet1997}. The syntax of Z-types, $\ztau$, and Z-expressions, $\zexp$, is given in Figure \ref{fig:zexp-syntax}. The only base cases in these inductive grammars are $\zwsel{\htau}$ and $\zwsel{\hexp}$, which identify the H-type or H-expression that is the current focus. All other forms correspond to the recursive forms in the syntax of H-types and H-expressions, and contain exactly one ``hatted'' subterm that identifies the subtree where the focus will be found. All other sub-terms are H-types or H-expressions. Taken together, every syntactically well-formed Z-type and Z-expression contains exactly one focused H-type or H-expression.

We write $\removeSel{\ztau}$ for the H-type constructed by removing the focus marker from the Z-type $\ztau$. This straightforward metafunction is defined as follows:
\begin{align*}
%\removeSel{(\zlsel{\htau})} & = \htau\\
\removeSel{(\zwsel{\htau})} & = \htau\\
%\removeSel{(\zrsel{\htau})} & = \htau\\
\removeSel{(\tarr{\ztau}{\htau})} & = \tarr{\removeSel{\ztau}}{\htau}\\
\removeSel{(\tarr{\htau}{\ztau})} & = \tarr{\htau}{\removeSel{\ztau}}
\end{align*}

Similarly, we write $\removeSel{\zexp}$ for the H-expression constructed by removing the focus marker from the Z-expression $\zexp$. The definition of this metafunction is analagous, so we leave it in the appendix for concision.
% \begin{align*}
% %\removeSel{(\zlsel{\hexp})} & = \hexp\\
% \removeSel{(\zwsel{\hexp})} & = \hexp\\
% %\removeSel{(\zrsel{\hexp})} & = \hexp\\
% \removeSel{(\zexp : \htau)} & = \removeSel{\zexp} : \htau\\
% \removeSel{(\hexp : \ztau)} & = \hexp : \removeSel{\ztau}\\
% \removeSel{(\hlam{x}{\zexp})} & = \hlam{x}{\removeSel{\zexp}}\\
% \removeSel{(\hap{\zexp}{\hexp})} & = \hap{\removeSel{\zexp}}{\hexp}\\
% \removeSel{(\hap{\hexp}{\zexp})} & = \hap{\hexp}{\removeSel{\zexp}}\\
% \removeSel{(\hadd{\zexp}{\hexp})} & = \hadd{\removeSel{\zexp}}{\hexp}\\
% \removeSel{(\hadd{\hexp}{\zexp})} & = \hadd{\hexp}{\removeSel{\zexp}}\\
% \removeSel{\hhole{\zexp}} &= \hhole{\removeSel{\zexp}}
% \end{align*}

\subsection{Action Semantics}\label{sec:actions}
\begin{figure}[t]
\hspace{-3px}$\arraycolsep=3pt\begin{array}{llcllll}
\mathsf{Action} & \alpha & ::= &
  \aMove{\delta} ~\vert~
  %\aSelect{\delta} ~\vert~
  \aDel ~\vert~
  %\aReplace{\htau} ~\vert~
  %\aReplace{\hexp} ~\vert~
  \aConstruct{\varphi} ~\vert~
  \aFinish\\
\mathsf{Direction} & \delta & ::= &
  \dChild ~\vert~
  \dParent ~\vert~
  \dNext ~\vert~
  \dPrev\\
\mathsf{Shape} & \varphi & ::= &
  \farr ~\vert~
  \fnum \\
& & \vert &
  \fasc ~\vert~
  \fvar{x} ~\vert~
  \flam{x} ~\vert~
  \fap ~\vert~
  \farg ~\vert~
  \fnumlit{n} ~\vert~
  \fplus
\end{array}$
%\textbf{Sort} & & & \textbf{Operational Form} & \textbf{Stylized Form} & \textbf{Description}\\
\caption{Syntax of actions.}
\label{fig:action-syntax}
\vspace{-8px}
\end{figure}

The syntax of \emph{actions}, $\alpha$, some of which involve \emph{directions}, $\delta$, or \emph{shapes}, $\varphi$, is given in Figure \ref{fig:action-syntax}. Actions are performed on Z-types and Z-expressions according to the \emph{action semantics} of Hazelnut, which is organized around three judgements:
\[\arraycolsep=10pt\begin{array}{ll}
%\textbf{Judgement Form} & \textbf{Description}\\
\performTyp{\ztau}{\alpha}{\ztau'} & \text{Performing $\alpha$ on $\ztau$ produces $\ztau'$}\\
\performSyn{\hGamma}{\zexp}{\htau}{\alpha}{\zexp'}{\htau'} & \text{Performing $\alpha$ on $\zexp$ when $\removeSel{\zexp}$ synthesizes type $\htau$}\\
& \text{produces $\zexp'$ such that $\removeSel{\zexp'}$ synthesizes type $\htau'$}\\
\performAna{\hGamma}{\zexp}{\htau}{\alpha}{\zexp'} & \text{Performing $\alpha$ on $\zexp$ when analyzing $\removeSel{\zexp}$ against $\htau$}\\
& \text{produces $\zexp'$, such that $\removeSel{\zexp'}$ can also be analyzed}\\
& \text{against $\htau$}
\end{array}\]

As suggested by the descriptions above, the action semantics maintains the following \emph{action sensibility} theorem:
\begin{theorem}[Action Sensibility] Both of the following hold:
\label{thrm:actsafe}
\begin{enumerate}
\item If $\performSyn{\hGamma}{\zexp}{\htau}{\alpha}{\zexp'}{\htau'}$ and
  $\hsyn{\hGamma}{\removeSel{\zexp}}{\htau}$ then
  $\hsyn{\hGamma}{\removeSel{\zexp'}}{\htau'}$.
\item If $\performAna{\hGamma}{\zexp}{\htau}{\alpha}{\zexp'}$ and
  $\hana{\hGamma}{\removeSel{\zexp}}{\htau}$ then
  $\hana{\hGamma}{\removeSel{\zexp'}}{\htau}$.
\end{enumerate}
\end{theorem}
In words, every action leaves the program in a semantically well-defined state. More specifically, the first clause of Theorem \ref{thrm:actsafe} establishes that actions performed on expressions that synthesize a type can only produce expressions that also synthesize some (possibly different) type. The second clause establishes that actions performed on expressions in analytic position (e.g. those under type ascriptions or in argument position, see above) can only produce expressions that can also be analyzed against the expected type.% Non-empty holes allow us to avoid top-down program construction becau but rather can construct fragments of the program inside a hole until ready to ``expose'' them to type analysis.

It is also useful to maintain a \emph{deterministic} action semantics, i.e. every well-defined action should produce a unique Z-type or Z-expression. Formally, this is stated as follows:
\begin{theorem}[Action Determinism] All of the following hold:
\label{thrm:actdet}
\begin{enumerate}
\item If $\performTyp{\ztau}{\alpha}{\ztau'}$ and $\performTyp{\ztau}{\alpha}{\ztau''}$ then $\ztau'=\ztau''$.
\item If $\hsyn{\hGamma}{\removeSel{\zexp}}{\htau}$ and
  $\performSyn{\hGamma}{\zexp}{\htau}{\alpha}{\zexp'}{\htau'}$ and
  $\performSyn{\hGamma}{\zexp}{\htau}{\alpha}{\zexp''}{\htau''}$ then
  $\zexp' = \zexp''$ and $\htau' = \htau''$.
% \item If all of

%   \begin{quote}
%     \begin{enumerate}
%     \item $\hsyn{\hGamma}{\removeSel{\zexp}}{\htau}$, and
%     \item $\performSyn{\hGamma}{\zexp}{\htau}{\alpha}{\zexp'}{\htau'}$, and
%     \item $\tcompat{\htau}{\htau'}$, and
%     \item either $\performAna{\hGamma}{\zexp}{\htau}{\alpha}{\zexp''}$ or
%       $\performAna{\hGamma}{\zexp}{\htau'}{\alpha}{\zexp''}$
%     \end{enumerate}
%   \end{quote}
%   hold, then $\zexp' = \zexp''$.
\item If $\hana{\hGamma}{\removeSel{\zexp}}{\htau}$ and
  $\performAna{\hGamma}{\zexp}{\htau}{\alpha}{\zexp'}$ and
  $\performAna{\hGamma}{\zexp}{\htau}{\alpha}{\zexp''}$ then $\zexp' =
  \zexp''$.
\end{enumerate}
\end{theorem}

In order to maintain determinism, we will need to supplement the definition of type compatibility above with a definition for \emph{type incompatibility}, $\tincompat{\htau}{\htau'}$. The key rule establishes that arrow types are incompatible with the $\tnum$ type:
\begin{subequations}
  % \begin{equation}
  %   \inferrule{
  %     \tincompat{\htau}{\htau'}
  %   }{
  %     \tincompat{\htau'}{\htau}
  %   }
  % \end{equation}
  \begin{equation}
    \inferrule{ }{
      \tincompat{\tnum}{\tarr{\htau_1}{\htau_2}}
    }
  \end{equation}
  % \begin{equation}
  %   \inferrule{
  %     \tincompat{\htau_1}{\htau_1'}
  %   }{
  %     \tincompat{\tarr{\htau_1}{\htau_2}}{\tarr{\htau_1'}{\htau_2'}}
  %   }
  % \end{equation}
  % \begin{equation}
  %   \inferrule{
  %     \tincompat{\htau_2}{\htau_2'}
  %   }{
  %     \tincompat{\tarr{\htau_1}{\htau_2}}{\tarr{\htau_1'}{\htau_2'}}
  %   }
  % \end{equation}
\end{subequations}
The remaining rules, given in the appendix, establish that type incompatibility is symmetric and covariant.
\subsubsection{Subsumption}

The action semantics includes a subsumption rule much like the one from the underlying semantics of H-expressions:
\begin{equation}
  \inferrule{
    \hsyn{\hGamma}{\removeSel{\zexp}}{\htau'}\\
    \performSyn{\hGamma}{\zexp}{\htau'}{\alpha}{\zexp'}{\htau''}\\
    \tcompat{\htau}{\htau''}
  }{
    \performAna{\hGamma}{\zexp}{\htau}{\alpha}{\zexp'}
  }
\end{equation}
In other words, if the expression synthesizes a type, then we defer to the synthetic action performance judgement, as long as it produces an expression that synthesizes a type compatible with the type provided for analysis. It is easy to see that this satisfies Theorem 1 by applying the IH and subsumption.

\subsubsection{Relative Movement} Movement actions change the focus but do not change the underlying H-type or H-expression (so action sensibility is easy to show for these rules as well.)

The rules for relative movement within Z-types are given below and should be self-explanatory:
\begin{subequations}
\begin{equation}
  \inferrule{ }{
    \performTyp{
      \zwsel{\tarr{\htau_1}{\htau_2}}
    }{
      \aMove{\dChild}
    }{
      \tarr{\zwsel{\htau_1}}{\htau_2}
    }
  }
\end{equation}
\begin{equation}
  \inferrule{ }{
    \performTyp{
      \tarr{\zwsel{\htau_1}}{\htau_2}
    }{
      \aMove{\dParent}
    }{
      \zwsel{\tarr{\htau_1}{\htau_2}}
    }
  }
\end{equation}
\begin{equation}
  \inferrule{ }{
    \performTyp{
      \tarr{{\htau_1}}{\zwsel{\htau_2}}
    }{
      \aMove{\dParent}
    }{
      \zwsel{\tarr{\htau_1}{\htau_2}}
    }
  }
\end{equation}
\begin{equation}
  \inferrule{ }{
    \performTyp{
      \tarr{\zwsel{\htau_1}}{{\htau_2}}
    }{
      \aMove{\dNext}
    }{
      {\tarr{\htau_1}{\zwsel{\htau_2}}}
    }
  }
\end{equation}
\begin{equation}
  \inferrule{ }{
    \performTyp{
      \tarr{{\htau_1}}{\zwsel{\htau_2}}
    }{
      \aMove{\dPrev}
    }{
      {\tarr{\zwsel{\htau_1}}{{\htau_2}}}
    }
  }
\end{equation}
% \begin{equation}
% \inferrule{
%   \performTyp{
%     \ztau
%   }{
%     \aMove{\delta}
%   }{
%     \ztau'
%   }
% }{
%   \performTyp{
%     \tarr{\ztau}{\htau}
%   }{
%     \aMove{\delta}
%   }{
%     \tarr{\ztau'}{\htau}
%   }
% }
% \end{equation}
% \begin{equation}
%   \inferrule{
%     \performTyp{
%       \ztau
%     }{
%       \aMove{\delta}
%     }{
%       \ztau'
%     }
%   }{
%     \performTyp{
%       \tarr{\htau}{\ztau}
%     }{
%       \aMove{\delta}
%     }{
%       \tarr{\htau}{\ztau}
%     }
%   }
% \end{equation}
\end{subequations}
% The final two rules above recurse into the zipper structure.

The rules for relative movement within Z-expressions are similar. Movement is type-independent, so we defer to an auxiliary judgement for both the analytic and synthetic judgements:
\begin{subequations}
\begin{equation}
\inferrule{
  \performMove{\zexp}{\aMove{\delta}}{\zexp'}
}{
  \performSyn{\hGamma}{\zexp}{\htau}{\aMove{\delta}}{\zexp'}{\htau}
}
\end{equation}
\begin{equation}
  \inferrule{
  \performMove{\zexp}{\aMove{\delta}}{\zexp'}
}{
  \performAna{\hGamma}{\zexp}{\htau}{\aMove{\delta}}{\zexp'}
}
\end{equation}
\end{subequations}
For concision, we show only the rules for ascription here:
\begin{subequations}
  \begin{equation}
    \label{r:movefirstchild}
  \inferrule{ }{
    \performTyp{
      \zwsel{\hexp : \htau}
    }{
      \aMove{\dChild}
    }{
      \zwsel{\hexp} : \htau
    }
  }
\end{equation}
\begin{equation}
  \label{r:moveparent}
  \inferrule{ }{
    \performTyp{
      \zwsel{\hexp} : \htau
    }{
      \aMove{\dParent}
    }{
      \zwsel{\hexp : \htau}
    }
  }
\end{equation}
\begin{equation}
  \inferrule{ }{
    \performTyp{
      \hexp : \zwsel{\htau}
    }{
      \aMove{\dParent}
    }{
      \zwsel{\hexp : \htau}
    }
  }
\end{equation}
\begin{equation}
  \label{r:movenextsib}
  \inferrule{ }{
    \performTyp{
      \zwsel{\hexp} : \htau
    }{
      \aMove{\dNext}
    }{
      \hexp : \zwsel{\htau}
    }
  }
\end{equation}
\begin{equation}
  \label{r:moveprevsib}
  \inferrule{ }{
    \performTyp{
      \hexp : \zwsel{\htau}
    }{
      \aMove{\dPrev}
    }{
      \zwsel{\hexp} : \htau
    }
  }
\end{equation}
\begin{equation}
\inferrule{
  \performTyp{
    \zexp
  }{
    \aMove{\delta}
  }{
    \zexp'
  }
}{
  \performTyp{
    \zexp : \htau
  }{
    \aMove{\delta}
  }{
    \zexp' : \htau
  }
}
\end{equation}
\begin{equation}
  \inferrule{
    \performTyp{
      \ztau
    }{
      \aMove{\delta}
    }{
      \ztau'
    }
  }{
    \performTyp{
      \hexp : \ztau
    }{
      \aMove{\delta}
    }{
      \hexp : \ztau'
    }
  }
\end{equation}
\end{subequations}
\subsubsection{Deletion} The $\aDel$ action replaces the selected subterm with an empty hole.

Again, the rule for Z-types is self-explanatory:
\begin{subequations}
\begin{equation}
  \inferrule{ }{
    \performTyp{
      \zwsel{\htau}
    }{
      \aDel
    }{
      \zwsel{\tehole}
    }
  }
\end{equation}
% \begin{equation}
%   \inferrule{
%     \performTyp{\ztau}{\aDel}{\ztau'}
%   }{
%     \performTyp{\tarr{\ztau}{\htau}}{\aDel}{\tarr{\ztau'}{\htau}}
%   }
% \end{equation}
% \begin{equation}
%   \inferrule{
%     \performTyp{\ztau}{\aDel}{\ztau'}
%   }{
%     \performTyp{\tarr{\htau}{\ztau}}{\aDel}{\tarr{\htau}{\ztau'}}
%   }
% \end{equation}
\end{subequations}

Deletion within a Z-expression is similarly straightforward:
\begin{subequations}
\begin{equation}
  \inferrule{ }{
    \performSyn{\hGamma}{\zwsel{\hexp}}{\htau}{\aDel}{\zwsel{\hehole}}{\tehole}
  }
\end{equation}
\begin{equation}
  \inferrule{ }{
    \performAna{\hGamma}{\zwsel{\hexp}}{\htau}{\aDel}{\zwsel{\hehole}}
  }
\end{equation}
%\end{subequations}
% The base case turns into a hole:
%\begin{subequations}
% \begin{equation}
% \inferrule{ }{
%   \performDel{\zwsel{\hexp}}{\hehole}
% }
% \end{equation}
% The rules for the recursive ascription case is shown below. The other recursive cases are analagous:
% \begin{equation}
%   \inferrule{
%     \performDel{\zexp}{\zexp'}
%   }{
%     \performDel{\zexp : \htau}{\zexp' : \htau}
%   }
% \end{equation}
% \begin{equation}
%   \inferrule{
%     \performTyp{\ztau}{\aDel}{\ztau'}
%   }{
%     \performDel{\hexp : \ztau}{\hexp : \ztau'}
%   }
% \end{equation}

\end{subequations}
\subsubsection{Construction} The construction actions, $\aConstruct{\varphi}$, are used to construct terms of a shape indicated by $\varphi$ into the program at or around the focus.

Again, let us begin with type actions. The $\aConstruct{\farr}$ action constructs an arrow type. The focused H-type becomes the argument type, and the focus is placed on an empty return type hole:
\begin{subequations}
  \begin{equation}
    \label{r:contarr}
  \inferrule{ }{
    \performTyp{
      \zwsel{\htau}
    }{
      \aConstruct{\farr}
    }{
      \tarr{\htau}{\zwsel{\tehole}}
    }
  }
\end{equation}

The $\aConstruct{\fnum}$ action replaces an empty Z-type hole with the $\tnum$ type:
  \begin{equation}
    \label{r:contnum}
  \inferrule{ }{
    \performTyp{
      \zwsel{\tehole}
    }{
      \aConstruct{\fnum}
    }{
      \zwsel{\tnum}
    }
  }
\end{equation}

% Construction proceeds recursively down the zipper:
%   \begin{equation}
%     \label{r:contarrL}
%   \inferrule{
%     \performTyp{\ztau}{\aConstruct{\varphi}}{\ztau'}
%   }{
%     \performTyp{
%       \tarr{\ztau}{\htau}
%     }{
%       \aConstruct{\varphi}
%     }{
%       \tarr{\ztau'}{\htau}
%     }
%   }
% \end{equation}
%   \begin{equation}
%     \label{r:contarrR}
%   \inferrule{
%     \performTyp{\ztau}{\aConstruct{\varphi}}{\ztau'}
%   }{
%     \performTyp{
%       \tarr{\htau}{\ztau}
%     }{
%       \aConstruct{\varphi}
%     }{
%       \tarr{\htau}{\ztau'}
%     }
%   }
% \end{equation}
\end{subequations}

\begin{subequations}

Moving on to expression actions, we start to see more interesting rules. The $\aConstruct{\fasc}$ action operates differently depending on whether the focused expression synthesizes a type or is being analyzed against a type. In the first case, the ascribed type is the synthesized type:
\begin{equation}
  \label{r:constructasc}
  \inferrule{ }{
    \performSyn{\hGamma}{\zwsel{\hexp}}{\htau}{\aConstruct{\fasc}}{\hexp : \zwsel{\htau}}{\htau}
  }
\end{equation}
In the second case, the ascribed type is the type provided for analysis:
\begin{equation}
  \inferrule{ }{
    \performAna{\hGamma}{\zwsel{\hexp}}{\htau}{\aConstruct{\fasc}}{\hexp : \zwsel{\htau}}
  }
\end{equation}

The $\aConstruct{\fvar{x}}$ action places the variable $x$ into the focused empty hole. If that hole is being asked to synthesize a type, then the result of the action synthesizes the type assigned to $x$ in the context:
\begin{equation}
  \label{r:conevar}
  \inferrule{ }{
    \performSyn{\hGamma, x : \htau}{\zwsel{\hehole}}{\tehole}{\aConstruct{\fvar{x}}}{\zwsel{x}}{\htau}
  }
\end{equation}
If the focused empty hole is being analyzed against a type that is inconsistent with the type assigned to $x$ by the context, $x$ is placed inside a hole:
\begin{equation}
 \label{r:conevar2}
  \inferrule{
    \tincompat{\htau}{\htau'}
  }{
    \performAna{\hGamma, x : \htau'}{\zwsel{\hehole}}{\htau}{\aConstruct{\fvar{x}}}{\hhole{\zwsel{x}}}
  }
\end{equation}
The rule above featured in the example in Section \ref{sec:example}.

Notice that no rule was necessary for the case where the hole was being analyzed against a type compatible with the variable's type, because this case is handled by the action subsumption rule.

The $\aConstruct{\flam{x}}$ action places a lambda term binding $x$ into an empty hole. If the focused empty hole is being asked to synthesize a type, then the result of the action is a lambda ascribed the type $\tarr{\tehole}{\tehole}$, with the focus in the argument type position:
\begin{equation}
  \label{r:conelamhole}
  \inferrule{ }{
    \performSyn
      {\hGamma}
      {\zwsel{\hehole}}
      {\tehole}
      {\aConstruct{\flam{x}}}
      {\hlam{x}{\hehole} : \tarr{\zwsel{\tehole}}{\tehole}}
      {\tarr{\tehole}{\tehole}}
  }
\end{equation}
The type ascription is necessary because lambda expressions do not synthesize a type. If the focused empty hole is being analyzed against an arrow type, then no ascription is necessary:
\begin{equation}
  \inferrule{ }{
    \performAna
      {\hGamma}
      {\zwsel{\hehole}}
      {\tarr{\htau_1}{\htau_2}}
      {\aConstruct{\flam{x}}}
      {\hlam{x}{\zwsel{\hehole}}}
  }
\end{equation}

If the focused empty hole is being analyzed against a type that is
incompatible with any arrow type, expressed in the premise as an arrow with
two holes, then a lambda ascribed the type $\tarr{\tehole}{\tehole}$
is inserted inside a hole, to maintain Theorem \ref{thrm:actsafe}:
\begin{equation}
  \inferrule{
    \tincompat{\htau}{\tarr{\tehole}{\tehole}}
  }{
    \performAna
      {\hGamma}
      {\zwsel{\hehole}}
      {\htau}
      {\aConstruct{\flam{x}}}
      {\hhole{
        \hlam{x}{\hehole} : \tarr{\zwsel{\tehole}}{\tehole}
      }}
  }
\end{equation}

The $\aConstruct{\fap}$ action applies the expression in focus to a hole. If the focused expression synthesizes a function type, then the rule is straightforward:
\begin{equation}
  \label{r:coneapfn}
  \inferrule{ }{
    \performSyn
      {\hGamma}
      {\zwsel{\hexp}}
      {\tarr{\htau_1}{\htau_2}}
      {\aConstruct{\fap}}
      {\hap{\hexp}{\zwsel{\hehole}}}
      {\htau_2}
  }
\end{equation}

If the focused expression synthesizes a hole type, then we can treat it as if it synthesized the $\tarr{\tehole}{\tehole}$ type, exactly as described in Sec. \ref{sec:holes}:
\begin{equation}
  \inferrule{ }{
    \performSyn
      {\hGamma}
      {\zwsel{\hexp}}
      {\tehole}
      {\aConstruct{\fap}}
      {\hap{\hexp}{\zwsel{\hehole}}}
      {\tehole}
  }
\end{equation}

Finally, if the focused expression synthesizes a type that is incompatible with an arrow type, then we must place that expression inside a hole to maintain Theorem \ref{sec:holes}:
\begin{equation}
  \inferrule{
    \tincompat{\htau}{\tarr{\tehole}{\tehole}}
  }{
    \performSyn
      {\hGamma}
      {\zwsel{\hexp}}
      {\htau}
      {\aConstruct{\fap}}
      {\hap{\hhole{\hexp}}{\zwsel{\hehole}}}
      {\tehole}
  }
\end{equation}

The $\aConstruct{\farg}$ action places the focused expression instead in the argument position of an application. Because the function position is always an empty hole in this situation, we only need a single rule:
\begin{equation}
  \inferrule{ }{
    \performSyn
      {\hGamma}
      {\zwsel{\hexp}}
      {\htau}
      {\aConstruct{\farg}}
      {\hap{\zwsel{\hehole}}{\hexp}}
      {\tehole}
  }
\end{equation}

The $\aConstruct{\fnumlit{n}}$ action places the number expression $\hnum{n}$ into an empty hole. If the focused hole is being asked to synthesize a type, then the rule is straightforward:
\begin{equation}
  \label{r:conenumnum}
  \inferrule{ }{
    \performSyn
      {\hGamma}
      {\zwsel{\hehole}}
      {\tehole}
      {\aConstruct{\fnumlit{n}}}
      {\zwsel{\hnum{n}}}
      {\tnum}
  }
\end{equation}
If the focused hole is being analyzed against a type that is incompatible with $\tnum$, then we must place the number expression inside a hole:
\begin{equation}
  \inferrule{
    \tincompat{\htau}{\tnum}
  }{
    \performAna
      {\hGamma}
      {\zwsel{\hehole}}
      {\htau}
      {\aConstruct{\fnumlit{n}}}
      {\hhole{\zwsel{\hnum{n}}}}
  }
\end{equation}

Finally, the $\aConstruct{\fplus}$ action constructs a plus expression with the focused expression as its first argument. If the focused expression synthesizes a type consistent with $\tnum$, then the rule is straightforward:
\begin{equation}
  \inferrule{
    \tcompat{\htau}{\tnum}
  }{
    \performSyn
      {\hGamma}
      {\zwsel{\hexp}}
      {\htau}
      {\aConstruct{\fplus}}
      {\hadd{\hexp}{\zwsel{\hehole}}}
      {\tnum}
  }
\end{equation}

Otherwise, we must place the focused expression inside a hole:
\begin{equation}
  \inferrule{
    \tincompat{\htau}{\tnum}
  }{
    \performSyn
      {\hGamma}
      {\zwsel{\hexp}}
      {\htau}
      {\aConstruct{\fplus}}
      {\hadd{\hhole{\hexp}}{\zwsel{\hehole}}}
      {\tnum}
  }
\end{equation}
\end{subequations}
Notice that we do not have an action that explicitly wraps an expression in a non-empty hole. These arise implicitly when an action that would not na\"ively satisfy Theorem \ref{thrm:actsafe} is performed (see Figure \ref{fig:first-example}.)

\subsubsection{Finishing}
The final action we will consider in Hazelnut is $\aFinish$, which finishes the focused non-empty hole.

If the focused non-empty hole appears in synthetic position, then it can always be finished:
\begin{subequations}
  \begin{equation}
    \label{r:finishana}
  \inferrule{
    \hsyn{\hGamma}{\hexp}{\htau'}
  }{
    \performSyn
      {\hGamma}
      {\zwsel{\hhole{\hexp}}}
      {\tehole}
      {\aFinish}
      {\zwsel{\hexp}}
      {\htau'}
  }
\end{equation}

If the focused non-empty hole appears in analytic position, then it can only be finished if the type synthesized for the wrapped expression is consistent with the type the hole is being analyzed against. This amounts to analyzing those contents against the provided type (by subsumption):
\begin{equation}
  \inferrule{
    \hana{\hGamma}{\hexp}{\htau}
  }{
    \performAna
      {\hGamma}
      {\zwsel{\hhole{\hexp}}}
      {\htau}
      {\aFinish}
      {\zwsel{\hexp}}
  }
\end{equation}
\end{subequations}

\subsubsection{Zipper Cases} The rules given so far handle the base cases, where the action has ``reached'' the focused expression. We also need to define the recursive cases, which propagate the action into the subtree where the focus appears. These rules follow the structure of the corresponding rules in the statics of H-expressions.

\begin{subequations}
For example, when the focus is in the expression position of an ascription, we use the analytic action performance judgement:
\begin{equation}
\inferrule{
  \performAna
    {\hGamma}
    {\zexp}
    {\htau}
    {\alpha}
    {\zexp'}
}{
  \performSyn
    {\hGamma}
    {\zexp : \htau}
    {\htau}
    {\alpha}
    {\zexp' : \htau}
    {\htau}
}
\end{equation}

When the focus is in the type position of an ascription, we must re-check the ascribed expression because the type might have changed (in practice, one would optimize this check to only occur if the type actually was changed):
\begin{equation}
\inferrule{
  \performTyp{\ztau}{\alpha}{\ztau'}\\
  \hana{\hGamma}{\hexp}{\removeSel{\ztau'}}
}{
  \performSyn
    {\hGamma}
    {\hexp : \ztau}
    {\removeSel{\ztau}}
    {\alpha}
    {\hexp : \ztau'}
    {\removeSel{\ztau'}}
}
\end{equation}

If the focus is in the body of a lambda expression, then we must use the analytic action performance rule:
\begin{equation}
\inferrule{
  \performAna
    {\hGamma, x : \htau_1}
    {\zexp}
    {\htau_2}
    {\alpha}
    {\zexp'}
}{
  \performAna
    {\hGamma}
    {\hlam{x}{\zexp}}
    {\tarr{\htau_1}{\htau_2}}
    {\alpha}
    {\hlam{x}{\zexp'}}
}
\end{equation}

There are two rules that handle the case where the focus is in the function position of an application, corresponding to the two application rules in the statics. Each involves rechecking the argument against the new function type:
\begin{equation}
  \inferrule{
    \hsyn{\hGamma}{\removeSel{\zexp}}{\htau_2}\\
    \performSyn
      {\hGamma}
      {\zexp}
      {\htau_2}
      {\alpha}
      {\zexp'}
      {\tarr{\htau_3}{\htau_4}}\\
    \hana{\hGamma}{\hexp}{\htau_3}
  }{
    \performSyn
      {\hGamma}
      {\hap{\zexp}{\hexp}}
      {\htau_1}
      {\alpha}
      {\hap{\zexp'}{\hexp}}
      {\htau_4}
  }
\end{equation}
\begin{equation}
  \inferrule{
    \hsyn{\hGamma}{\removeSel{\zexp}}{\htau_2}\\
    \performSyn
      {\hGamma}
      {\zexp}
      {\htau_2}
      {\alpha}
      {\zexp'}
      {\tehole}\\
    \hana{\hGamma}{\hexp}{\tehole}
  }{
    \performSyn
      {\hGamma}
      {\hap{\zexp}{\hexp}}
      {\htau_1}
      {\alpha}
      {\hap{\zexp'}{\hexp}}
      {\tehole}
  }
\end{equation}

Similarly, there are two rules that handle the case where the focus is in the argument position:
\begin{equation}
  \inferrule{
    \hsyn{\hGamma}{\hexp}{\tarr{\htau_2}{\htau}}\\
    \performAna
      {\hGamma}
      {\zexp}
      {\htau_2}
      {\alpha}
      {\zexp'}
  }{
    \performSyn
      {\hGamma}
      {\hap{\hexp}{\zexp}}
      {\htau}
      {\alpha}
      {\hap{\hexp}{\zexp'}}
      {\htau}
  }
\end{equation}
\begin{equation}
  \inferrule{
    \hsyn{\hGamma}{\hexp}{\tehole}\\
    \performAna
      {\hGamma}
      {\zexp}
      {\tehole}
      {\alpha}
      {\zexp'}
  }{
    \performSyn
      {\hGamma}
      {\hap{\hexp}{\zexp}}
      {\tehole}
      {\alpha}
      {\hap{\hexp}{\zexp'}}
      {\tehole}
  }
\end{equation}

The rules for the addition operator follow from the statics directly:
\begin{equation}
  \inferrule{
    \performAna
      {\hGamma}
      {\zexp}
      {\tnum}
      {\alpha}
      {\zexp'}
  }{
    \performSyn
      {\hGamma}
      {\hadd{\zexp}{\hexp}}
      {\tnum}
      {\alpha}
      {\hadd{\zexp'}{\hexp}}
      {\tnum}
  }
\end{equation}
\begin{equation}
  \inferrule{
    \performAna
      {\hGamma}
      {\zexp}
      {\tnum}
      {\alpha}
      {\zexp'}
  }{
    \performSyn
      {\hGamma}
      {\hadd{\hexp}{\zexp}}
      {\tnum}
      {\alpha}
      {\hadd{\hexp}{\zexp'}}
      {\tnum}
  }
\end{equation}

Finally, if the focus is inside a non-empty hole, we special case the situation where the action results in a doubly-nested empty hole, $\hhole{\hehole}$, to eliminate the nesting (given our current action semantics, only the delete action can cause this form to arise and the form $\hhole{\hhole{\zexp}}$ cannot arise):
\begin{equation}
  \inferrule{
    \hsyn{\hGamma}{\removeSel{\zexp}}{\htau}\\
    \performSyn
      {\hGamma}
      {\zexp}
      {\htau}
      {\alpha}
      {\zexp'}
      {\htau'}\\
    \zexp' \neq \zwsel{\hehole}
  }{
    \performSyn
      {\hGamma}
      {\hhole{\zexp}}
      {\tehole}
      {\alpha}
      {\hhole{\zexp'}}
      {\tehole}
  }
\end{equation}
\begin{equation}
  \inferrule{
    \hsyn{\hGamma}{\removeSel{\zexp}}{\htau}\\
    \performSyn
      {\hGamma}
      {\zexp}
      {\htau}
      {\alpha}
      {\zwsel{\hehole}}
      {\tehole}\\
  }{
    \performSyn
      {\hGamma}
      {\hhole{\zexp}}
      {\tehole}
      {\alpha}
      {\zwsel{\hehole}}
      {\tehole}
  }
\end{equation}

\end{subequations}

\section{Mechanization}
\label{sec:mech}\label{sec:mt}
% !TEX root = hazelnut-popl17.tex
% So far, we have given an overview of the most important judgements and
% rules in the semantics of Hazelnut, and stated the critical metatheorem,
% Sensibility, and several auxiliary ``checks''. In a few cases, we have
% informally sketched out why these metatheorems will hold. 

In order to
formally establish that our design meets our stated objectives, we have
mechanized the semantics and metatheory of Hazelnut as described above using the Agda proof
assistant \cite{norell:thesis} (also see the Agda Wiki, hosted
at \url{http://wiki.portal.chalmers.se/agda/}.) This development is available in the supplemental material. The mechanization also includes the  extension to Hazelnut described in Sec. \ref{sec:extending}.

The documentation includes a more detailed discussion of the technical
representation decisions that we made. The main idea is standard: we encode
each judgement as a dependent type. The rules defining the judgements
become the constructors of this type, and derivations are terms of these
type. This is a rich setting that allows proofs to take advantage of
pattern matching on the shape of derivations, closely matching standard
on-paper proofs. No proof automation was used, because the proof structure
itself is likely to be interesting to researchers who plan to build upon
our work.

We adopt Barendregt's convention for bound
variables \cite{urban}. Hazelnut's semantics does not need substitution, so
we do not need to adopt more sophisticated encodings
(e.g. \cite{lh09unibind,Pouillard11}.)



\section{Implementation}
\label{sec:impl}
\subsection{Implementation Concepts}

\begin{figure}
\centering
\includegraphics[width=0.75\columnwidth]{impl-overview}
\caption{}
\label{fig:impl-overview}
\end{figure}











\todo{make figure take up only one column}\todo{change terminology in figure}\todo{revise}
The key question that must be answered for any implementation strategy is: how do we model a stream of actions from a user? Let us assume that these actions are chosen (using some input device, preferably, a keyboard) from some ``palette'' that never presents the user with actions that are not semantically well-defined, according to the action semantics defined earlier.
%In a traditional editor, the input from the user is a stream of characters, and there are no guarantees that at any point that the program is syntactically well-formed, so the designer leaves editing as an .
%In contrast, in a structure editor, the input from the user is a stream of operations.
As such, each new action will ``atomically'' generate a new Z-expression. 
This insight leads us to conclude that a natural way to implement this editor would be using event-based Functional Reactive Programming~\cite{Wan:2000:FRP:349299.349331} (FRP).
Figure~\ref{fig:FRP} illustrates the concept of an FRP-based implementation of a  structure editor organized like Hazelnut.
The input from the user is a stream of actions.  Each action results in a change to the underlying abstract model (i.e., a new Z-expression is created after each action.)
Each model change results in an updated \emph{view} which is then presented to the user.  The user can then consider this new view when they choose a new action as input.

\subsection{HZ}
We explore the concepts presented in the paper in HZ, our implementation of Hazelnut.
In order to reach a wide audience, we decided to implement HZ in the web browser.
In order to take advantage of all the benefits of FRP, we chose to implement HZ using OCaml\footnote{https://ocaml.org/}, the \texttt{js\_of\_ocaml} compiler\footnote{http://ocsigen.org/js\_of\_ocaml/} and the OCaml React library\footnote{http://erratique.ch/software/react}.

At the time of the writing of this paper, our implementation of HZ includes encodings of Z-expression as presented in this paper.
We consider this ZExp to be our model. 
HZ renders the model as a string embedded in HTML.
Currently we support only the delete action.  Other actions are currently under development. We anticipate having a substantially more functional implementation by the time this work is presented (our focus thusfar has been on the metatheory.) 
The work-in-progress code as well as directions for how to compile and run it can be found here: \url{https://github.com/hazelgrove/impl-tfp16}.

A substantially simpler system that we developed while exploring the ideas that led to Hazelnut can be found at the following URL:
\url{http://www.cs.cmu.edu/~comar/nestedpairs/}.


\section{Related Work}\label{sec:rw}
%\subsection{Structure Editors}

Structured editing has been recognized as a way to avoid the possibility of syntax errors for decades.  An early example is the
The Cornell Program Synthesizer~\cite{teitelbaum_cornell_1981}, first published in 1981.
The synthesizer generator~\cite{Reps:1984:SG:390010.808247} allows the user to create an attribute-grammar specification that then can be used to generate a structured editor.
CENTAUR~\cite{Borras:1988:CS:64140.65005} produces a language specific environment from a user defined formal specification of a language. Barista \cite{ko_barista:_2006} is a modern take on the same basic concept.%These early systems were developed  of the systems are rooted in the type-theoretic tradition.

Novice programmers have been a common target for structure editors. For example,
GNOME\cite{garlan_gnome:_1984} was developed to teach programming to undergraduates.
Scratch~\cite{Resnick:2009:SP:1592761.1592779} is a structure editor targeted at children ages 8 to 16.
Touchdevelop \cite{tillmann_touchdevelop:_2011} incorporates a structure editor for programming on touch-based devices, and is used to teach high school students.
Alice~\cite{Conway:2000:ALL:332040.332481} is a 3-D programming language with an integrated structure editor for teaching novice CS undergraduate students. These are largely drag-and-drop user interfaces with a limited action model and an unclear semantics.

Not all structure editors are for educational purposes. For example,
mbeddr \cite{voelter_mbeddr:_2012} is an extensible C-based Programming Language and IDE (nominally, for programming embedded systems.)
mbeddr is build on top of the commercial JetBrains MPS framework for constructing structure editors.
Another popular approach is to bring elements of structured editing into a traditional editor.
Codelets \cite{oney_codelets:_2012} uses structured editing to add interactive documentation and examples in an editor.
Our previous work on Graphite~\cite{Omar:2012:ACC:2337223.2337324} allows developers to associate structured editing interfaces called  \emph{palettes} with types. Graphite is integrated into a text-based program editor (Eclipse.)

Agda and Idris are two dependently typed languages that attempt to simulate a structured editor from within a rich text editor (e.g. Emacs.) These systems also have notions of holes and use types to guide the user toward filling these holes. These  systems are also, to our knowledge, not formally well-defined but rather exist only as part of system implementations.

Perhaps the systems most similar in spirit to Hazelnut are Lamdu~\cite{lamdu} and Unison~\cite{unison}. Like Hazelnut, these are both statically typed functional language editors. In both cases, the language is similar to Haskell. In Lamdu, the editor uses structure editing to enable Live Programming, where the code is always being executed as it is being written.

Our work differs from all of these in that we begin with a formal editor calculus and build from there, rather than starting with an implementation and leaving many of the formal details formally unspecified. For example, while Lamdu has many interesting features, there is no theoretical basis presented for their work -- it is a rather large body of Haskell code with an unclear (and indeed, often somewhat perplexing, in our experience) action model. Unison is also a rather large body of Haskell code, though its action model appears superficially more similar to ours. We maintain what we believe to be a stronger action sensibility invariant than Unison (i.e. in Unison, one must construct expressions from the outside-in.) These systems are rich sources of interesting ideas, however -- there is room enough for many different approaches in this (re-)emerging space.

TODO: cite http://cseweb.ucsd.edu/~lerner/pb.html

%Drag-and-drop / for novices: lots of examples, e.g. Alice and others
%
%Contemporary: Lamdu, MPS/Mbeddr, TouchDevelop
%
%Hybrid: Cyrus' active code completion paper

%\subsection{Refactoring Models}
%(Michael, can you fill this section out?)

%\subsection{Formal Editor Models}
%Need to do a search to see what else has been done...

\section{Discussion \& Conclusion}
\label{sec:future}
This paper presented Hazelnut, a type theoretic structure editor calculus. Our aim is to take a principled approach to its design by formally specifying its semantics, providing strong metatheoretic guarantees, mechanizing its semantics and metatheory in Agda and implementing it using the concepts of  functional reactive programming. As of this submission, we have achieved reasonable confidence in the formal system presented above, and have transitioned our focus toward the mechanization and implementation efforts. By the time of presentation, we anticipate having complete or nearly complete versions of these.

\subsection{Future Work}
Hazelnut is, obviously, a very limited language at its core. So the most obvious avenue for future work is to increase the expressive power of this language. Our plan is to simultaneously maintain a mechanization and implementation (following, for example, Standard ML) as we proceed, ultimately producing the first large-scale, formally verified bidirectionally typed language codesigned with a type-aware editor. It may be that certain language features are unnecessary given a sufficiently advanced type-aware structure editor (e.g. SML's \texttt{open}?), while other features may only be practical with editor support. We intend to use Hazelnut and derivative systems thereof as a platform for rigorously exploring such questions.

There are various aspects of the editor model that we have not yet formalized. For example, our action model does not consider how actions are actually entered using, for example, key combinations or chords. It also did not provide any specific model of how available actions will be determined for presentation to the user. In practice, we would want also to rank available actions in some reasonable manner (perhaps based on usage data gathered from other users or code repositories.)

Another research direction is in exploring how types can be used to control the presentation of expressions in the editor. For example, following our approach in a textual setting on \emph{type-specific languages} (TSLs), it should be possible to have the type that an expression is being analyzed against define alternative display forms and interaction modes \cite{TSLs}.

Finally, we did not consider any aspects of \emph{collaborative programming}, such as a packaging system, a differencing algorithm for use in a source control system, support for multiple simultaneous focii for different users, and so on. These are all interesting avenues for future work.


On the theoretical side, the notion of having one of many possible holes in a term in focus has a very strong intuitive connection
  to the proof theoretic notion of focusing \cite{Simmons11tr}. Beyond just
  the name, both seem to involve, in some sense, a search through the space of possible
  ways to finish a derivation. We intend to explore this connection to see
  if it's coincidental or more meaningful and welcome insights in this regard.

We already discussed a connection to gradual typing \cite{Siek06a}. We hope to explore this connection more thoroughly. In particular, it may be possible to better support exploratory and live programming by allowing even programs with holes in them to execute as long as those holes are only in the type portions, by deferring to the semantics given in work on gradual typing.

It may also be possible to give a dynamics to incomplete expressions. Prior work on staged evaluation suggests that there may be a connection to modal logic, viewing holes as quantifying over all possible terms that may fill them \cite{DBLP:journals/jacm/DaviesP01}. In developing a dynamic semantics, we will also need to handle terms like $\hhole{\hehole}$ and
$\hhole{\hhole{\hexp}}$. In our semantics given here, we eliminated them as they came up in a somewhat \emph{ad hoc} manner. We have not yet
explored an equational theory for terms with holes, but intend to once our
formalization effort is more mature.

\begin{quote}
In any case, these are but steps toward more graphical program-description
systems, for we will not forever stay confined to mere strings of symbols.

--- Marvin Minsky, Turing Award lecture
\end{quote}% We recommend abbrvnat bibliography style.
\clearpage
\bibliographystyle{abbrvnat}

% The bibliography should be embedded for final submission.

\bibliography{../research}
%\softraggedright
%P. Q. Smith, and X. Y. Jones. ...reference text...

\end{document}
