%% Cyrus: We took a look at the dynamics. Overall it seems like the
%% right idea. We noticed:

%% Done: 1) The stepping rules are non-deterministic (i.e. you can
%% step the right or left of e1 + e2 in any order). Might be useful to
%% make them deterministic.

%% Done (mostly): 2) The premises that have disjunctions in them could
%% be broken out into two rules -- this would take a little more space
%% but follows the usual conventions more closely.

%% Done (mostly): 3) We need to add the ``ceil'' forms to the grammar
%% of \dot{e} (you just used e) and give them a static semantics.

%% TODO: 4) We need to figure out (the analogs of) canonical forms,
%% preservation and progress -- I guess we had decided on defining a
%% declarative statics to do that. Ian has started to prove the
%% correspondence (sans the ceil forms).

%% #4 seems like the most important next step.

\begin{figure}[htbp]
  \centering

  \judgbox{\hexp~\textsf{value}}{H-Expression $\hexp$ is a closed value}
  \begin{mathpar}
    \Infer{V-num}
          { }
          {\hnum{n}~\textsf{value}}
    \and
    \Infer{V-lam}
        {}
        {\hlam{x}{\hexp}~\textsf{value}}
  \end{mathpar}
  
  \caption{Value forms}
  \label{fig:judg-value}
\end{figure}

\begin{figure}[htbp]
  \centering

  \judgbox{\hexp~\textsf{final}}{H-Expression $\hexp$ is final}
  \begin{mathpar}
    \Infer{F-val}
          {\hexp~\textsf{value}}
          {\hexp~\textsf{final}}
    \and
    \Infer{F-filled}
          {\hexp~\textsf{final}}
          {\hhole{\hexp}~\textsf{final}}
    \and
    \Infer{F-unfilled}
          { }
          {\hhole{}~\textsf{final}}
    \and
    \Infer{F-indet}
        {\hexp~\textsf{indet}}
        {\hindet{\hexp}~\textsf{final}}
  \end{mathpar}
  
  \caption{Final forms}
  \label{fig:judg-value}
\end{figure}

\begin{figure}[htbp]
  \centering

  \judgbox{\hexp~\textsf{indet}}{H-Expression~$\hexp$ is indeterminate}
  \begin{mathpar}
    \Infer{I-plus$_1$}
          { \hexp_1~\textsf{final} \\
            \hexp_2~\textsf{final} \\
            \hexp_1 \ne \hnum{n_1}
          }
          {\hadd{e_1}{e_2}~\textsf{indet}}
    \and
    \Infer{I-plus$_2$}
          { \hexp_1~\textsf{final} \\
            \hexp_2~\textsf{final} \\
            \hexp_2 \ne \hnum{n_2}
          }
          {\hadd{\hexp_1}{\hexp_2}~\textsf{indet}}
    \and
    \Infer{I-app}
          { \hexp_1~\textsf{final} \\
            \hexp_2~\textsf{final} \\
            \hexp_1 \ne \hlam{x}{\hexp_1'}
          }
          {\hap{\hexp_1}{\hexp_2}~\textsf{indet}}
    \and
  \end{mathpar}
  
  \caption{Indeterminate forms}
  \label{fig:judg-value}
\end{figure}

\begin{figure}[htbp]
  \centering

  \judgbox{\hexp_1 \longrightarrow \hexp_2}{H-Expression $\hexp_1$ steps to~$\hexp_2$}
  \begin{mathpar}
    \Infer{S-plus$_1$}
          { \hexp_1 \longrightarrow \hexp_1' }
          { \hadd{\hexp_1}{\hexp_2} \longrightarrow \hadd{\hexp_1'}{\hexp_2} }

    \Infer{S-plus$_2$}
          { \hexp_1~\textsf{final}
            \\
            \hexp_2 \longrightarrow \hexp_2' }
          { \hadd{\hexp_1}{\hexp_2} \longrightarrow \hadd{\hexp_1}{\hexp_2'} }

    \Infer{S-plus$_3$}
          { n_1 + n_2 = n_3 }
          { \hadd{\hnum{n_1}}{\hnum{n_2}} \longrightarrow \hnum{n_3} }

   \Infer{S-plus$_4$}
          { \hexp_1~\textsf{final} ~~ ~~ ~~
            \hexp_2~\textsf{final} 
            \\\\
            \left(\hexp_1 \ne \hnum{n_1} \vee \hexp_2 \ne \hnum{n_2}\right)
          }
          { \hadd{\hexp_1}{\hexp_2} \longrightarrow \hindet{\hadd{\hexp_1}{\hexp_2}} }

    \Infer{S-ap$_1$}
          { \hexp_1 \longrightarrow \hexp_1' }
          { \hap{\hexp_1}{\hexp_2} \longrightarrow \hap{\hexp_1'}{\hexp_2} }

    \Infer{S-ap$_2$}
          { \hexp_2 \longrightarrow \hexp_2'
            \\
            \hexp_1~\textsf{final}
          }
          { \hap{\hexp_1}{\hexp_2} \longrightarrow \hap{\hexp_1}{\hexp_2'} }

    \Infer{S-ap$_3$}
          { \hexp_2~\textsf{final} }
          { \hap{\hlam{x}{\hexp_1}}{\hexp_2} \longrightarrow [\hexp_2/x]\hexp_1 }

   \Infer{S-ap$_4$}
          { \hexp_1~\textsf{final} ~~ ~~ ~~
            \hexp_2~\textsf{final}
            \\\\
            \hexp_1 \ne \hlam{x}{\hexp_1'} ~~ ~~ ~~ ~~ ~~ ~~ ~~ { }
          }
          { \hap{\hexp_1}{\hexp_2} \longrightarrow \hindet{\hap{\hexp_1}{\hexp_2}} }

  \end{mathpar}
  
  \caption{Small-step operational semantics.}
  \label{fig:judg-value}
\end{figure}

% Q: Are we running programs with ascriptions?
% No. We erase the ascriptions before running. Ascriptions below are
%
% Q: Are the ascriptions needed to type-check the programs?
% No. The ascriptions are needed to check the programs algorithmically, not to derive typing derivations.
% (Q: How do ascriptions, holes, and typing derivations interact?)
%
% Q: Does it ever make sense to run open programs? (i.e., run programs with ``free variables''?)
%  -- Yes, but only in the sense that we can bind holes to variables,
%     ascribe them types, and (attempt to) compute with them.
%
% Example: (where '?' means "empty hole")
%  let x = ? : int -> int in
%  let y = ? : int in
%  x y : int
% ==erase==>
%  let x = ? in
%  let y = ? in
%  x y
% -->
%  let y = ? in
%  ? y
% -->
%  ? ?
% --> 
%  (indet ? ?)

We extend the syntax of H-Expressions as follows:
\[
\hexp ::= \cdots~|~\hindet{\hexp}
\]

Here are some things that we want to prove:

\textbf{Def}~(Ascription erasure). 
\\
$\herase{\hexp}$ is the same as $\hexp$, but without its type ascriptions.

\textbf{Conjecture}~(Bidirectional implies declarative).
\\
(i) If $\hsyn{\hGamma}{\hexp}{\htau}$ then $\hGamma \vdash \herase{\hexp} : \htau$.
\\
(ii) If $\hana{\hGamma}{\hexp}{\htau}$ then $\hGamma \vdash \herase{\hexp} : \htau$.

\textbf{Conjecture}~(Substitution).
\\
If $\hGamma, x : \tau_x \vdash \hexp : \htau$
\\
and $\hexp'~\textsf{final}$
\\
and $\hGamma \vdash \hexp' : \htau_x$
\\
then $\hGamma \vdash \hexp[\hexp' / x] : \htau$

% Q: Do we need to enforce that hexp' is final?

\textbf{Conjecture}~(Canonical forms)
\\
If $\cdot \vdash \hexp : \htau$
\\
and $\hexp~\textsf{final}$ then
\begin{itemize}
\item if $\hexp = \hindet{\hexp'}$ then $\hexp'~\textsf{indet}$ and $\cdot \vdash \hexp' : \htau$
%\marginnote{Since $\hexp'~\textsf{indet}$, the conjecture applies to $\hexp'$.}
\item else:
\begin{itemize}
\item if $\htau = \tnum$ then exists $\hnum{n}$ such that $\hexp = \hnum{n}$.
\item else if $\htau = \tarr{\htau_1}{\htau_2}$ then exists $x$ and $\hexp'$ such that $\hexp = \hlam{x}{\hexp'}$
\item else if $\htau = \tehole$ then either:
\begin{itemize}
\item $\hexp = \hehole$, or
\item exists $\htau'$ and $\hexp'$ such that $\hexp = \hhole{\hexp'}$, $\hexp'~\textsf{final}$ and $\cdot \vdash \hexp' : \htau'$
\end{itemize}
\end{itemize}
\end{itemize}

\textbf{Conjecture}~(Progresss).
\\
If $\cdot \vdash \hexp_1 : \htau$
\\
then either $\hexp_1~\textsf{final}$
\\
or exists $\hexp_2$ such that $\hexp_1 \longrightarrow \hexp_2$.

\textbf{Conjecture}~(Preservation).
\\
If $\cdot\vdash \hexp_1 : \htau$ 
\\
and $\hexp_1 \longrightarrow \hexp_2$
\\
then $\cdot \vdash \hexp_2 : \htau$
