% !TEX root = hazelnut-popl17.tex
In the previous section, we gave an overview of the most important rules in
the semantics of Hazelnut, and stated the important theorems. In a few
cases, we sketched out why these theorems will hold.

In order to formally verify that our design meets the stated objectives, we
are preparing a formalization of the grammar, the judgements given above
and the metatheory in the proof assistant Agda \cite{norell:thesis}. We
refer readers unfamiliar with Agda to the Agda Wiki, hosted
at \url{http://wiki.portal.chalmers.se/agda/}.

Our formalization of Hazelnut is under development at
\url{http://github.com/hazelgrove/agda-tfp16}. At the time of submission, we 
have completed much of the initial groundwork, but the proofs of 
Theorem \ref{thrm:actsafe} and Theorem \ref{thrm:actdet} (and a few  
lemmas that we elided here) are under construction.  

The documentation 
includes a more detailed discussion of the technical representation
decisions made. The core idea of our formalization is to encode each judgement as a
dependent type. The rules of the judgements become the constructors of the
type, and derivations of theorems values of the type. This is a rich
setting that allows proofs to take advantage of pattern matching on the
shape of derivations, closely matching on-paper proofs of similar
properties.

The formalization differs from the calculus defined here in a few small 
ways. Most interestingly, instead of giving separate movement rules for each 
form of Z-expression, we abstract
over different corresponding forms that happen to have the same arity, e.g. 
additions and applications. Formalizing this intuition
reduces the number of cases we need to consider 
somewhat, but more importantly allows us to write a slightly more
general calculus -- it will be easier to extend Hazelnut with more 
interesting language features with this generic infrastructure in place.

Because the metatheory in this paper is largely concerned with the statics
of Hazelnut rather than its dynamics, we adopt Barendregt's convention for
bound variables and avoid substitution entirely \cite{urban}. Future
iterations will need a more mature technique for reasoning about
binding---likely de Bruijn indices or abstract binding
trees \cite{lh09unibind,Pouillard11}. 