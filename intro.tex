% !TEX root = hazelnut-popl17.tex

Programmers typically interact with meaningful programs only
indirectly, by editing text that is first parsed according to a textual
syntax and then typechecked according to a static semantics. This
indirection has practical benefits, to be sure -- text editors and other
text-based tools benefit from decades of development effort. However, this indirection through text also
introduces some fundamental complexity into the programming process.

First, it requires that programmers learn the subtleties of the textual
syntax (e.g. operator precedence.) This is particularly 
challenging for novices \cite{Altadmri:2015:MCI:2676723.2677258,stefik2013empirical,Marceau:2011:VGT:2016911.2016921}, and even experienced programmers frequently make mistakes \cite{7548903,jones2006developer,Marceau:2011:VGT:2016911.2016921}.

Second, many sequences of characters do not correspond to meaningful
programs. This complicates the design of program editors and other interactive programming tools. 
%In particular, program
%editors must contend with meaningless text on a regular basis. 
In a dataset  
gathered by Yoon and Myers consisting of 1460 hours of Eclipse edit logs  \cite{6883030}, 44.2\% of
edit states were syntactically malformed. Some additional
percentage of edit states were well-formed but ill-typed (the dataset was
not complete enough to determine the exact percentage.) Collectively, we refer to these edit states as \emph{meaningless edit states}, because they are not given static or dynamic meaning by the language definition. 
%Some of
%these meaningless edit states arose because the programmer was in the midst
%of a sequence of edits that left the edit state transiently meaningless,
%while others arose because the programmer made a logical mistake. 
As a consequence, it is difficult
to design useful language-aware editor services, e.g. syntax
highlighting~\cite{sarkar2015impact}, type-aware code
completion~\cite{Mooty:2010:CCC:1915084.1916348,Omar:2012:ACC:2337223.2337324},
and refactoring services \cite{mens2004survey}. Editors must either disable
these editor services when they encounter meaningless edit states or
deploy \emph{ad hoc} heuristics, e.g. by using whitespace to guess the intent \cite{DBLP:conf/oopsla/KatsJNV09,DBLP:conf/sle/JongeNKV09}.

These complications have motivated a long line of research
into \emph{structure editors}, i.e. program editors where every edit state
corresponds to a program structure \cite{teitelbaum_cornell_1981}.\footnote{Structure editors are also variously known as \emph{structured editors}, \emph{structural editors}, \emph{syntax-directed editors} and \emph{projectional editors}.}

Most structure editors are \emph{syntactic structure editors}, i.e. the
edit state corresponds to a syntax tree with \emph{holes} that stand for branches of
the tree that have yet to be constructed, and the edit actions are
context-free tree transformations. For example, Scratch is a syntactic
structure editor that has achieved success as a tool for teaching children
how to program \cite{Resnick:2009:SP:1592761.1592779}. %Every well-formed edit state is meaningful according to the semantics of Scratch (in particular, it dynamically continues past erroneous terms.)

% Scratch does not impose any static requirements , i.e. every well-formed edit state is trivially statically meaningful. 
Researchers have also 
 designed  syntactic structure editors for more sophisticated languages with non-trivial  
static type systems. Contemporary examples include \texttt{mbeddr}, a structure editor for a C-like
language \cite{voelter_mbeddr:_2012}, TouchDevelop, a structure editor for an
object-oriented language \cite{tillmann_touchdevelop:_2011}, and Lamdu, a structure 
editor for a functional language similar to Haskell \cite{lamdu}. Each of
these editors presents an innovative user interface, but the non-trivial
type and binding structure of the underlying language complicates its
design. The problem is that syntactic structure editors do not assign static meaning to every edit state -- they guarantee only that every edit state corresponds to  a 
syntactically well-formed tree. These editors must also either selectively disable editor services that require an understanding of the semantics of the program being written, or deploy \emph{ad hoc} heuristics.


This paper develops a principled solution to this problem. We introduce
Hazelnut, a \emph{typed structure editor} based on a bidirectionally typed
lambda calculus extended to assign static meaning to expressions and types
with {holes}, which we call \textbf{H-expressions}
and \textbf{H-types}. Hazelnut's formal \emph{action semantics} maintains
the invariant that every edit state is a statically meaningful
(i.e. well-H-typed) H-expression with a single superimposed \emph{cursor}. We
call H-expressions and H-types with a cursor \textbf{Z-expressions}
and \textbf{Z-types} (so prefixed because our encoding follows
Huet's \emph{zipper} pattern \cite{JFP::Huet1997}.)

Na\"ively, enforcing an injunction on ill-H-typed edit states would force
programmers to construct programs in a rigid ``outside-in'' manner. For
example, the programmer would often need to construct the outer function
application form before identifying the intended function. To address this
problem, Hazelnut leaves newly constructed expressions \emph{inside} a hole
if the expression's type is inconsistent with the expected type. This
meaningfully defers the type consistency check until the expression inside
the hole is \emph{finished}. In other words, holes appear both at the
leaves and at the internal nodes of an H-expression that remain under
construction.

The remainder of this paper is organized as follows:
\begin{itemize}[itemsep=0px,partopsep=2px,topsep=2px]
  \item We begin in Sec.  \ref{sec:example} with two examples of edit
    sequences to develop the reader's intuitions.

  \item We then give a detailed overview of Hazelnut's semantics and
  metatheory, which has been mechanized using the Agda proof assistant, in
  Sec.  \ref{sec:hazel}.

  \item Hazelnut is a {foundational} calculus, i.e. a calculus that
  language and editor designers are expected to extend with higher level
  constructs. We extend Hazelnut with binary sum types in
  Sec.  \ref{sec:extending} to demonstrate how Hazelnut's rich metatheory
  guides one such extension.

  \item In Sec.  \ref{sec:impl}, we briefly describe how Hazelnut's action
  semantics lends itself to efficient implementation as a functional
  reactive program. Our reference implementation is written
  using the OCaml \lstinline{React} library and \lstinline{js_of_ocaml}.

  \item In Sec.  \ref{sec:rw}, we summarize related work. In particular,
  much of the technical machinery needed to handle type holes coincides
  with machinery developed for gradual type systems. Similarly,
  expression holes can be interpreted as the closures of contextual modal
  type theory, which, by its correspondence with contextual modal logic,
  suggests logical foundations for the system.

  \item We conclude in Sec.  \ref{sec:future} by summarizing our vision of
  a principled science of structure editor design rooted in type theory,
  and suggest a number of future directions.
\end{itemize}
The supplemental material can be accessed from:
\[\text{\url{http://cs.cmu.edu/~comar/hazelnut-popl17/}} \]
% 1) a typeset listing of Hazelnut's rules
%in definitional order; 2) the formalization of Hazelnut in Agda; and 3) our
%reference implementation of Hazelnut, both in source form and pre-compiled
%to JavaScript for use in a web browser.\todo{how are we going to distribute supplemental material? maybe a GitHub URL like the one for AE?}
